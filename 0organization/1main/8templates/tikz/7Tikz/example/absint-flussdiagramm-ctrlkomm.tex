\input{"./2ProjectSetup".tex}%
\input{"\DenKrLayoutMainRootDir/2includes/packages/preamble_pre".tex}%
\documentclass[tikz,fontsize=11pt,class=scrbook]{standalone}% I.e. the content from \input{"\DenKrLayoutMainRootDir/2layout/tikz_standalone/preamble_1_class".tex}%
\input{"\DenKrLayoutBaseRootDir/tikz_standalone/1TikzStandalonePicIncludeThis".tex}%
\DenKrTikzStandalonePre%
%
%
%
\newcommand\nodedistance{0.5cm}%
\newcommand\innabst{0.8em}% Abstand zwischen Text und Zellenrand
%
\let\breite\undefined%
\newlength\breite%
\settowidth{\breite}{\large \textbf{Interfacename}}%
\setlength\breite{\breite+\innabst+\innabst+1em}%
%
%
\begin{tikzpicture}[%
	scale = \tikzpicturescale,%
	auto,%
	node distance=\nodedistance%
]%
\newcommand{\cnt}{2}%
%
%
% \bfseries%
%
%\draw[color=slightblue,fill] (-0.5,-4.5) rectangle (10.5,0.5);
%
	\node(ctrlkommstart)%
		[note]%
		{Thread zur Kommunikation\\ mit Controller};%
%
	\node(ctrlconnection)%
		[io,%
		below=1*\nodedistance of ctrlkommstart]%
		{Kontaktiere AI-Controller};%
	\draw[arrow](ctrlkommstart)--(ctrlconnection);%
%
	\node(startthread)%
		[process,%
		below=1*\nodedistance of ctrlconnection]%
		{Starte Thread zum\\ Empfangen von Nachrichten};%
	\draw[arrow](ctrlconnection)--(startthread);%
	\node(threadsplit)[shape=coordinate,%
		below=3*\nodedistance of startthread]{};%
%
	\node(mutterthread)%
		[note,%
		left=7*\nodedistance of threadsplit]%
		{Thread zum Senden\\ von Nachrichten};%
	\node(mutterthread_above)[shape=coordinate,%
		above=1.3*\nodedistance of mutterthread]{};%
	\draw[arrow](mutterthread_above)--%
		node[pos=0,anchor=north west]{Mutterthread}%
		(mutterthread);%
	\draw[connect]($(startthread.south)+(-1cm,0)$)|-(mutterthread_above);%
%
	\node(sendports)%
		[io,%
		below=1.5*\nodedistance of mutterthread]%
		{Sende Ports\\ an AI-Controller};%
	\draw[arrow](mutterthread)--(sendports);%
%
	\node(wlanmonneigh)%
		[process,%
		below=1*\nodedistance of sendports]%
		{WLAN-Monitor:\\ Ermittle WLAN Stationen\\ in Reichweite};%
	\draw[arrow](sendports)--(wlanmonneigh);%
%
	\node(sendwlanneigh)%
		[io,%
		below=1*\nodedistance of wlanmonneigh]%
		{Sende WLAN Nachbarn\\ an AI-Controller};%
	\draw[arrow](wlanmonneigh)--(sendwlanneigh);%
%
%
\node(initiallu)[shape=coordinate]at($(sendwlanneigh.south-|wlanmonneigh.west)+(-0.8*\nodedistance,-0.5*\nodedistance)$);%
\node(initialro)[shape=coordinate]at($(sendports.north-|wlanmonneigh.east)+(0.8*\nodedistance,0.75*\nodedistance)$);%
\path[draw,thick,loosely dotted](initiallu)|-(initialro)|-(initiallu);%
\node()[cleartext,anchor=south west,inner sep=3pt,rotate=-90]%
	at(initialro)%
	{Initial-Setup Präambel};
%
%
\node(sendloopstart)[shape=coordinate,below=of sendwlanneigh]{};%
	\draw[connect](sendwlanneigh)--(sendloopstart);%
%
	\node(semwaitsend)%
		[process,%
		below=1*\nodedistance of sendloopstart]%
		{\lstinline[language=C_var,basicstyle=\ttfamily]%
		{sem_wait(&sem_send);}};%
	\draw[arrow](sendloopstart)--(semwaitsend);%
%
	\node(checkctrlconnect)%
		[interthread,%
		below=1*\nodedistance of semwaitsend]%
		{Prüfe Controller Verbindung};%
	\draw[arrow](semwaitsend)--(checkctrlconnect);%
%
% 	\node(switchmodeflag)%
% 		[decision,%
% 		below=1*\nodedistance of checkctrlconnect]%
% 		{\ico{FLAG\_CHECK(}\\%
% 		\ico{CHAN\_SWITCH\_MODE)}};%
% 	\draw[arrow](checkctrlconnect)--(switchmodeflag);%
%
	\node(ifctrlconnected)%
		[decision,%
		below=1.25*\nodedistance of checkctrlconnect]%
		{Controller\\ getrennt?};%
	\draw[arrow](checkctrlconnect)--(ifctrlconnected);%
% 	\draw[arrow](switchmodeflag)--%
% 		node[pos=0,anchor=north east]{Nein}%
% 		(ifctrlconnected);%
%
\node(reconnectctrl)[shape=coordinate,xshift=-1.5em]at(ctrlconnection-|checkctrlconnect.west){};%
	\draw[arrow](reconnectctrl)--(ctrlconnection);%
	\draw[connect](ifctrlconnected.west)-|%
		node[pos=0,anchor=south east,align=right]{Ja}%
		(reconnectctrl);%
%
	\node(shmem)%
		[interthread,%
		below=1.5*\nodedistance of ifctrlconnected]%
		{ShMem Zugriff};%
	\draw[arrow](ifctrlconnected)--%
		node[pos=0,anchor=north east]{Nein}%
		(shmem);%
%
	\node(adhocisset_kommdest)%
		[cleartext,%
		right=3*\nodedistance of shmem]%
		{\textbf{\lipsaround{IT-switch1}}};%
	\draw[arrowkomm,<-](shmem)--(adhocisset_kommdest);%
%
	\node(switchmsg)%
		[mux,%
		text width=5.15cm,%
		below=1*\nodedistance of shmem]%
		{Nachrichten-Typ zu senden};%
	\draw[arrow](shmem)--(switchmsg);%
%
	\node(msgswitch_rts)%
		[muxwahl2,%
		anchor=north,%
		xshift=1.5cm]%
		at(switchmsg.south)%
		{};%
	\node(msgswitch_rts_text)%
		[cleartext,anchor=north,inner sep=3pt]%
		at(msgswitch_rts.north)%
		{Ready to\\ switch};%
	\node(msgswitch_etc)%
		[muxwahl2,%
		anchor=north,%
		xshift=-1.5cm]%
		at(switchmsg.south)%
		{};%
	\node(msgswitch_etc_text)%
		[cleartext,anchor=center,inner sep=1em]%
		at(msgswitch_etc)%
		{\textbf{\ldots}};%
%
	\node(collectinfo)%
		[processsplit=3,%
		rectangle split part align={center, left, left},%
		below=1*\nodedistance of msgswitch_etc]%
		{\nodepart{one}ggf. Informationen\\ akquirieren%
		\nodepart{two}z.B. WLAN Nachbarn%
		\nodepart{three}z.B. Station Ports};%
	\draw[arrow](msgswitch_etc)--(collectinfo);%
%
	\node(genericpack)%
		[process,%
		anchor=north,%
		text width=5.1cm,%
		yshift=-\nodedistance]%
		at(switchmsg.south|-collectinfo.south)%
		{Generisches Packen\\ der Nachricht};%
	\draw[arrow](msgswitch_rts.south)--(msgswitch_rts.south|-genericpack.north);%
	\draw[arrow](collectinfo.south)--(collectinfo.south|-genericpack.north);%
%
	\node(sendmsg)%
		[io,%
		below=1*\nodedistance of genericpack]%
		{Nachricht an AI-Controller senden};%
	\draw[arrow](genericpack)--(sendmsg);%
%
%
	\node(adhocisset_kommdest2)%
		[cleartext,%
		right=3*\nodedistance of sendmsg]%
		{\textbf{\lipsaround{C-switch1}}};%
	\draw[arrowwait,->](sendmsg)--(adhocisset_kommdest2);%
%
	\node(shmempost_send)%
		[interthread,%
		below=\nodedistance of sendmsg]%
		{ShMem Freigabe};%
	\draw[arrow](sendmsg)--(shmempost_send);%
%
\node(sendloopend)[shape=coordinate]%
at($(shmempost_send.south west)+(-3.5cm,-0.5*\nodedistance)$){};%
	\draw[connect](shmempost_send)|-(sendloopend);%
	\draw[arrow](sendloopend)|-(sendloopstart);%
%
%
%
%
%
%
%
%
%
%
%
%
%
%
%
%
%
%
%
	\node(listenthreadstart)%
		[note,%
		right=7*\nodedistance of threadsplit]%
		{Thread zum \uline{\ien{blocked}}\\ Empfangen von Nachrichten};%
	\node(listenthreadstart_above)[shape=coordinate,%
		above=1.3*\nodedistance of listenthreadstart]{};%
	\draw[arrow](listenthreadstart_above)--(listenthreadstart);%
	\draw[connect]($(startthread.south)+(1cm,0)$)|-(listenthreadstart_above);%
%
	\node(sockblock)%
		[io,%
		below=2*\nodedistance of listenthreadstart]%
		{Erwarte Nachricht vom Controller\\%
		(blocked read am \ico{socket})};%
	\draw[arrow](listenthreadstart)--(sockblock);%
%
%
	\node(channelswitched_kommdest2)%
		[cleartext,%
		left=3*\nodedistance of sockblock]%
		{\textbf{\lipsaround{C-switch2}}};%
	\draw[arrowwait,<-](sockblock)--(channelswitched_kommdest2);%
%
	\node(sockdisco)%
		[decision,%
		below=1*\nodedistance of sockblock]%
		{Controller\\ getrennt?};%
	\draw[arrow](sockblock)--(sockdisco);%
%
	\node(ctrldisconnect_shmem)%
		[interthread,%
		anchor=north,%
		xshift=-0.5cm,%
		yshift=-2*\nodedistance]%
		at(sockdisco.west)%
		{Vermerke getrennte\\ Verbindung im ShMem};%
	\draw[arrow](sockdisco.west)-|%
		node[pos=0,anchor=south east]{Ja}%
		(ctrldisconnect_shmem);%
%
	\node(listenstop)%
		[startstop,%
		below=1*\nodedistance of ctrldisconnect_shmem]%
		{Exit\\ Thread};%
	\draw[arrow](ctrldisconnect_shmem)--(listenstop);%
%
\node(disconnecthelp)[shape=coordinate]at($(checkctrlconnect.east)!.5!(ctrldisconnect_shmem.west)$){};%
\path[draw,arrowkomm,->](ctrldisconnect_shmem.west)-|(disconnecthelp)|-(checkctrlconnect.east);
%
	\node(shmemlisten)%
		[interthread,%
		anchor=north,%
		yshift=-2*\nodedistance]%
		at(sockdisco.south|-listenstop.south east)%
		{ShMem Zugriff};%
	\draw[arrow](sockdisco)--%
		node[pos=0,anchor=north west]{Nein}%
		(shmemlisten);%
%
	\node(channelswitched_kommdest)%
		[cleartext,%
		left=3*\nodedistance of shmemlisten]%
		{\textbf{\lipsaround{IT-switch2}}};%
	\draw[arrowkomm,->](shmemlisten)--(channelswitched_kommdest);%
%
	\node(ctrlmsgread)%
		[processsplit=3,%
		below=2*\nodedistance of shmemlisten]%
		{\nodepart{one}Nachricht auslesen%
		\nodepart{two}ShMem Header beschreiben%
		\nodepart{three}ShMem gemäß\\ Nachricht präparieren};%
	\draw[arrow](shmemlisten)--(ctrlmsgread);%
%
	\node(sempost_mainsynch)%
		[process,below=2*\nodedistance of ctrlmsgread]%
		{\lstinline[language=C_var,basicstyle=\ttfamily]%
		{sem_post(&mainsynch);}};%
	\draw[arrow](ctrlmsgread)--(sempost_mainsynch);%
%
	\node(shmempost_listen)%
		[interthread,%
		below=2*\nodedistance of sempost_mainsynch]%
		{ShMem Freigabe};%
	\draw[arrow](sempost_mainsynch)--(shmempost_listen);%
%
\node(listenloop)[shape=coordinate,xshift=1*\nodedistance,yshift=-0.5*\nodedistance]%
	at(shmempost_listen.south-|sockblock.east){};%
	\draw[connect](shmempost_listen.south)|-(listenloop);%
	\draw[arrow](listenloop)|-($(sockblock.north)+(0,1*\nodedistance)$);%
%
%
%
%
%
%
%
%
%
%
%
%
%
%
%
%
%
%
%
%
%
%
%
%
%
%
\normalfont%
%
\end{tikzpicture}%
%
%
\DenKrTikzStandalonePost%