\input{"./2ProjectSetup".tex}%
\input{"\DenKrLayoutMainRootDir/2includes/packages/preamble_pre".tex}%
\documentclass[tikz,fontsize=11pt,class=scrbook]{standalone}% I.e. the content from \input{"\DenKrLayoutMainRootDir/2layout/tikz_standalone/preamble_1_class".tex}%
\input{"\DenKrLayoutBaseRootDir/tikz_standalone/1TikzStandalonePicIncludeThis".tex}%
\DenKrTikzStandalonePre%
%
%
%
\newcommand\nodedistance{0.5cm}%
%
\let\innabst\undefined%
\newlength\innabst%
\setlength\innabst{0.8em}% Abstand zwischen Text und Zellenrand
%
\let\ifaceheight\undefined%
\newlength\ifaceheight%
\settoheight{\ifaceheight}{\textbf{wlan0}}%
\setlength\ifaceheight{\ifaceheight+2\innabst}%
%
\let\ifacewidth\undefined%
\newlength\ifacewidth%
\settowidth{\ifacewidth}{\textbf{Adresse B00}}%
\setlength\ifacewidth{\ifacewidth+\innabst-0.5em}%
%
\let\ifacesep\undefined%
\newlength\ifacesep%
\setlength\ifacesep{0.3cm}%
%
\let\abifacewidth\undefined%
\newlength\abifacewidth%
\setlength\abifacewidth{2\ifacewidth+6\ifacesep}%
%
\let\abifaceheight\undefined%
\newlength\abifaceheight%
\setlength\abifaceheight{2.5\ifaceheight}%
%
\let\abipcheight\undefined%
\newlength\abipcheight%
\setlength\abipcheight{2\abifaceheight}%
%
\let\abipcwidth\undefined%
\newlength\abipcwidth%
\setlength\abipcwidth{\abifacewidth+3\ifacesep}%
%
\newcommand\roundcorneramount{2mm}%
%
% Define the layers to draw the diagram
\pgfdeclarelayer{background}%
\pgfdeclarelayer{cable}%
% \pgfdeclarelayer{foreground}%
\pgfsetlayers{background,main,cable}%,foreground}
%
%
\begin{tikzpicture}[%
	scale = 1.0,%Könnte das Makro \tikzpicturescale sein. Siehe in makros.tex
	% (Sollte eigentlich übergeben werden)(Beachte Möglichkeiten der Übergabe;
	% standalone mode=tex vs mode=buildnew)
	auto,%
	node distance=\nodedistance,%
]%
%
\tikzstyle{iface}=[%
	rectangle,%
% 	rounded corners=\roundcorneramount,%
% 	minimum width=\breiteEins,%
	minimum height=\ifaceheight,%
	text width=\ifacewidth,%
	text centered,%
	align=center,%
	draw=black,%
% 	inner sep=0.5*\innabst,%
	fill=yellow!50%
]%
%
\definecolor{abifacecolor}{RGB}{250,250,200}
\definecolor{abifacebackcolor}{RGB}{180,180,100}
\tikzstyle{abiface}=[%
	rectangle,%
% 	rounded corners=\roundcorneramount,%
	minimum width=\abifacewidth,%
	minimum height=\abifaceheight,%
	text centered,%
	align=center,%
	draw=black,%
	inner sep=0.5*\innabst,%
	fill=abifacecolor%
% 	middle color=yellow!20,
% 	right color=gray!20,
]%
%
\tikzstyle{abifaceoben}=[%
	abiface,%
	anchor=south,%
	shape=rectangleRoundedAnchorExact,%
	rectangle with rounded corners north west=15pt,%
	rectangle with rounded corners south west=0pt,%
	rectangle with rounded corners north east=15pt,%
	rectangle with rounded corners south east=0pt,%
]%
%
\tikzstyle{abifaceunten}=[%
	abiface,%
	anchor=north,%
	shape=rectangleRoundedAnchorExact,%
	rectangle with rounded corners north west=0pt,%
	rectangle with rounded corners south west=15pt,%
	rectangle with rounded corners north east=0pt,%
	rectangle with rounded corners south east=15pt,%
]%
%
\tikzstyle{abifaceobenback}=[%
	abifaceoben,%
	fill=abifacebackcolor,%
]%
%
\tikzstyle{abifaceuntenback}=[%
	abifaceunten,%
	fill=abifacebackcolor,%
]%
%
\tikzstyle{abipc}=[%
	rectangle,%
% 	rounded corners=\roundcorneramount,%
	minimum width=\abipcwidth,%
	minimum height=\abipcheight,%
	text centered,%
	align=center,%
	draw=black,%
	inner sep=0.5*\innabst,%
	fill=gray!20,%
	blur shadow={
% 		color=black,%
		shadow xshift=1mm,
		shadow yshift=1mm,
		shadow scale=1.0,%
		shadow opacity=50,%
		shadow blur radius=1em,%
		shadow blur steps=20,%
	}%
]%
%
\tikzstyle{station}=[%
	abipc,%
% 	rounded corners=\roundcorneramount,%
	minimum width=0.8\abipcwidth,%
	minimum height=\abipcheight,%
]%
%
\tikzstyle{stationiface}=[%
	abiface,%
% 	rounded corners=\roundcorneramount,%
	minimum width=0.8\abifacewidth,%
	minimum height=\abifaceheight,%
]%
%
\tikzstyle{stationifaceoben}=[%
	stationiface,%
	anchor=south,%
	shape=rectangleRoundedAnchorSurround,%
	rectangle with rounded corners north west=10pt,%
	rectangle with rounded corners south west=0pt,%
	rectangle with rounded corners north east=10pt,%
	rectangle with rounded corners south east=0pt,%
]%
%
\tikzstyle{stationifaceunten}=[%
	stationiface,%
	anchor=north,%
	shape=rectangleRoundedAnchorSurround,%
	rectangle with rounded corners north west=0pt,%
	rectangle with rounded corners south west=10pt,%
	rectangle with rounded corners north east=0pt,%
	rectangle with rounded corners south east=10pt,%
]%
%
\tikzstyle{stationifaceobenback}=[%
	stationifaceoben,%
	fill=abifacebackcolor,%
]%
%
\tikzstyle{stationifaceuntenback}=[%
	stationifaceunten,%
	fill=abifacebackcolor,%
]%
%
%
\bfseries%
%
	\node(hostA)[abipc,anchor=south]at(0,0){};%
	\node(hostAhead)[anchor=north]at(hostA.north){\LARGE Host
	\lipsaround{A}};%
%
	\node(hostANet)[shape=coordinate]at(hostA.south){};%
	\ExtrudeOutRoundedTop{hostANet}{up}{left}{abifaceoben}{abifaceobenback}%
		{abifacecolor}{abifacebackcolor}{15pt}%
	\node(hostANethead)[anchor=north]at(hostANetfront.north){\Large Netzwerk
	Schnittstellen};%
%
	\node(adrA1)[iface,anchor=south east]%
		at($(hostANetfront.south)+(-0.5\ifacesep,0)$){Adresse A1};%
	\node(adrA2)[iface,right=\ifacesep of adrA1]{Adresse A2};%
%
%
%
%
	\node(hostB)[abipc,anchor=south]at(10.5,0){};%
	\node(hostBhead)[anchor=north]at(hostB.north){\LARGE Host
	\lipsaround{B}};%
%
	\node(hostBNet)[shape=coordinate]at(hostB.south){};%
	\ExtrudeOutRoundedTop{hostBNet}{up}{left}{abifaceoben}{abifaceobenback}%
		{abifacecolor}{abifacebackcolor}{15pt}%
	\node(hostBNethead)[anchor=north]at(hostBNetfront.north){\Large Netzwerk
	Schnittstellen};%
%
	\node(adrB1)[iface,anchor=south east]%
		at($(hostBNetfront.south)+(-0.5\ifacesep,0)$){Adresse B1};%
	\node(adrB2)[iface,right=\ifacesep of adrB1]{Adresse B2};%
%
%
%
\let\connectdist\undefined%
\newlength\connectdist%
\setlength\connectdist{0.5cm}%
	\node(connectA1_1)[shape=coordinate]at($(adrA1.south)+(0,-2cm)$){};
	\node(connectA1_2)[shape=coordinate]at($(connectA1_1)+(0,-\connectdist)$){};
	\node(connectB1_1)[shape=coordinate]at($(adrB1.south)+(0,-2cm)$){};
	\node(connectB1_2)[shape=coordinate]at($(connectB1_1)+(0,-\connectdist)$){};
	\node(connectA2_1)[shape=coordinate]at($(adrA2.south)+(0,-4cm)$){};
	\node(connectA2_2)[shape=coordinate]at($(connectA2_1)+(0,-\connectdist)$){};
	\node(connectB2_1)[shape=coordinate]at($(adrB1.south)+(0,-4cm)$){};
	\node(connectB2_2)[shape=coordinate]at($(connectB2_1)+(0,-\connectdist)$){};
%
%
%
%
%
\begin{pgfonlayer}{cable}%
	\draw[cable](adrA1.south)--($(adrA1.south)+(0,-5cm)$);%
	\draw[cable](adrA2.south)--($(adrA2.south)+(0,-5cm)$);%
	\draw[cable](adrB1.south)--($(adrB1.south)+(0,-5cm)$);%
	\draw[cable](adrB2.south)--($(adrB2.south)+(0,-5cm)$);%
%
%
%
	\draw[pathMPTCP1](connectA1_1)--node[fill=white,anchor=south,yshift=0.25em,pos=0.5]%
		{\texttt{(Initiales Verbindungs Setup)}}%
		(connectB1_1);
	\draw[pathMPTCP1](connectB1_2)--(connectA1_2);
	\draw[pathMPTCP2](connectA2_1)--node[fill=white,anchor=south,xshift=-0.5em,yshift=0.25em,pos=0.5]%
		{\texttt{(Zusätzliches Subflow Setup)}}%
		(connectB2_1);
	\draw[pathMPTCP2](connectB2_2)--(connectA2_2);
\end{pgfonlayer}%
%
%
%
%
%
%
%
%
\normalfont%
\node(bboxtop)[shape=coordinate]at%
	($(current bounding box.north)+(0,1mm)+(0,1em)$){};%
\node(bboxright)[shape=coordinate]at%
	($(current bounding box.east)+(1mm,0)+(1em,0)$){};%
\node(bboxbottom)[shape=coordinate]at%
	($(current bounding box.south)+(0,0mm)+(0,0em)$){};%
\node(bboxleft)[shape=coordinate]at%
	($(current bounding box.west)+(1mm,0)+(-1em,0)$){};%
\path[use as bounding box]%
	(bboxtop)-|(bboxright)|-(bboxbottom)-|(bboxleft)|-(bboxtop);%
%
\end{tikzpicture}%
%
%
\DenKrTikzStandalonePost%