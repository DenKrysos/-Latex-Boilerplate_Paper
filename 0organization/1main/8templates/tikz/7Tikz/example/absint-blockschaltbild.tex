\input{"./2ProjectSetup".tex}%
\input{"\DenKrLayoutMainRootDir/2includes/packages/preamble_pre".tex}%
\documentclass[tikz,fontsize=11pt,class=scrbook]{standalone}% I.e. the content from \input{"\DenKrLayoutMainRootDir/2layout/tikz_standalone/preamble_1_class".tex}%
\input{"\DenKrLayoutBaseRootDir/tikz_standalone/1TikzStandalonePicIncludeThis".tex}%
\DenKrTikzStandalonePre%
%
%
%
\newcommand\nodedistance{1.5cm}%
\newcommand\innabst{0.8em}% Abstand zwischen Text und Zellenrand
%
\let\breite\undefined%
\newlength\breite%
\settowidth{\breite}{\large \textbf{Interfacename}}%
\setlength\breite{\breite+\innabst+\innabst+1em}%
%
% Define the layers to draw the diagram
\pgfdeclarelayer{background}%
%\pgfdeclarelayer{foreground}
\pgfsetlayers{background,main}%,foreground}
%
\begin{tikzpicture}[%
	scale = \tikzpicturescale,%
	auto,%
	node distance=\nodedistance%
]%
\newcommand{\cnt}{2}%
%
\definecolor{critred}{RGB}{255,70,40}%
\definecolor{entryred}{RGB}{255,50,50}%
\definecolor{slightblue}{RGB}{230,230,255}%
\definecolor{processblue}{RGB}{180,180,255}%
\definecolor{processblue2}{RGB}{90,180,250}%
%
\tikzstyle{background}=[%
	rectangle,%
	rounded corners,%
	minimum width=3cm,%
	minimum height=1cm,%
	text centered,%
	draw=black,%
	fill=slightblue%
]%
\tikzstyle{process1} = [%
	rectangle,%
	minimum width=3cm,%
	minimum height=1cm,%
	text centered,%
	align=center,%
	inner sep=\innabst,%
	draw=black,%
	fill=processblue%
]%
\tikzstyle{processsplit} = [%
	minimum width=3cm,%
	%minimum height=1cm,%
	text centered,%
	align=left,%
	inner sep=0.5*\innabst,%
	draw=black,%
	rectangle split,%
	rectangle split parts=#1,%
	rectangle split part fill={processblue,processblue2},%
    rectangle split every empty part={},% delete existing height, depth and width
	rectangle split empty part height=0.1ex,%
	rectangle split empty part depth=0ex,%
% 	shade,%
% 	shading=axis,%
% 	top color=white,%
% 	bottom color=processblue,%
% 	shading angle=45,%
]%
%
%
\begin{pgfonlayer}{main}%
\bfseries%
%
%
	\node(absint)%
		[process1]%
		{\ico{AbsInt}};%
	\node(wlanmon)%
		[process1,%
		below=of absint]%
		{WLAN-Monitor};%
	\node(netman)%
		[process1,%
		left=of wlanmon]%
		{netman};%
	\node(kanalwechsel)%
		[process1,%
		right=of wlanmon]%
		{Seamless Kanal-Wechsel};%
	\node(netzwerk)%
		[processsplit=4,%
		anchor=north east,%
		xshift=-1.25cm,%
		yshift=-\nodedistance]%
		at(netman.south east)%
		{\nodepart{one}Netzwerk%
		\nodepart{two}ARP-Tabellen\\ füllen%
		\nodepart{three}ARP-Tabellen\\ auslesen%
		\nodepart{four}Direkte Port\\ Nachbarn ermitteln};%
	\node(interfaces)%
		[processsplit=4,%
		anchor=north west,%
		xshift=0.25*\nodedistance]%
		at(netzwerk.north east)%
		{\nodepart{one}Interfaces%
		\nodepart{two}Liefert:\\ Ports\\ IPv4 Adressen%
		\nodepart{three}Selektion der\\ WLAN Interfaces%
		\nodepart{four}Prüft Existenz\\ eines Ports};%
	\node(wlanpakete)%
		[processsplit=3,%
		anchor=north west,%
		xshift=0.75*\nodedistance]%
		at(interfaces.north east)%
		{\nodepart{one}WLAN Pakete
		\nodepart{two}Parsing roher\\ WLAN Pakete
		\nodepart{three}Extraktion von:\\ Sender\\ Empfangene\\ \ \ Signalstärke};%
	\node(wlanverbindung)
		[processsplit=3,%
		anchor=north west,%
		xshift=0.75*\nodedistance]%
		at(wlanpakete.north east)%
		{\nodepart{one}WLAN Verbindung
		\nodepart{two}Einstellen von:\\ SSID\\ Kanal
		\nodepart{three}Ad-hoc\\ \ starten/stoppen};%
	\node(kanalbelegung)
		[processsplit=2,%
		anchor=north,%
		xshift=0.375*\nodedistance,%
		yshift=-0.25*\nodedistance]%
		at(interfaces.south east)%
		{\nodepart{one}WLAN Kanal\\ Statistik
		\nodepart{two}Belegung\\ überlappungsfreier\\ Kanäle durch\\ Datenverkehr};%
	\node(ctrlkomm)
		[processsplit=3,%
		anchor=north,%
		xshift=0.375*\nodedistance]%
		at(kanalbelegung.north east -| wlanpakete.south east)%
		{\nodepart{one}AI Controller\\ Kommunikation
		\nodepart{two}Daten an\\ Controller senden
		\nodepart{three}Anweisung von\\ Controller empfangen};%
%
%
%
	\draw[arrow](absint.south)--(wlanmon.north);
	\node(netmanabove)%
		[shape=coordinate]%
		at($(netman.north)+(0,0.5*\nodedistance)$)
		{};%
	\draw[connect]($(absint.south)+(-0.75cm,0)$)|-(netmanabove);
	\draw[arrow](netmanabove)--(netman.north);
	\node(kanalwechselabove)%
		[shape=coordinate]%
		at($(kanalwechsel.north)+(0,0.5*\nodedistance)$)
		{};%
	\draw[connect]($(absint.south)+(0.75cm,0)$)|-(kanalwechselabove);
	\draw[arrow](kanalwechselabove)--(kanalwechsel.north);
%
	\node(netzwerkabove)%
		[shape=coordinate]%
		at($(netzwerk.north)+(0,0.5*\nodedistance)$)
		{};%
	\draw[connect]($(netman.south)+(-0.5cm,0)$)|-(netzwerkabove);
	\draw[arrow](netzwerkabove)--(netzwerk.north);
%
	\node(interfacesabove1)%
		[shape=coordinate]%
		at($(interfaces.north)+(-0.75cm,0.5*\nodedistance)$)
		{};%
	\node(interfacesabove2)%
		[shape=coordinate]%
		at($(interfaces.north)+(0.75cm,0.5*\nodedistance)$)
		{};%
	\draw[connect]($(netman.south)+(0.5cm,0)$)|-(interfacesabove1);
	\draw[arrow](interfacesabove1)--(interfaces.north -| interfacesabove1);
%
	\draw[connect](wlanmon.south -| interfacesabove2)--(interfacesabove2);
	\draw[arrow](interfacesabove2)--(interfaces.north -| interfacesabove2);
	\draw[arrow](wlanmon.south -| kanalbelegung.north)--(kanalbelegung.north);
	\node(wlanpaketeabove)%
		[shape=coordinate]%
		at($(wlanpakete.north)+(-0.75cm,0.33*\nodedistance)$)
		{};%
	\draw[connect](wlanmon.south -| wlanpakete.north west)|-(wlanpaketeabove);
	\draw[arrow](wlanpaketeabove)--(wlanpakete.north -| wlanpaketeabove);
%
	\node(ctrlkommabove1)%
		[shape=coordinate]%
		at($(wlanpakete.north east)+(0.2*\nodedistance,0.67*\nodedistance)$)
		{};%
	\node(ctrlkommabove2)%
		[shape=coordinate]%
		at($(wlanverbindung.north west)+(-0.2*\nodedistance,0.5*\nodedistance)$)
		{};%
%
	\draw[connect]($(wlanmon.south -| wlanpakete.north west)+
		(0.33*\nodedistance,0)$)
			|-(ctrlkommabove1);%
	\draw[arrow](ctrlkommabove1)--(ctrlkomm.north -| ctrlkommabove1);%
	\draw[connect](kanalwechsel.south)|-(ctrlkommabove2);%
	\draw[arrow](ctrlkommabove2)--(ctrlkomm.north -| ctrlkommabove2);%
	\draw[arrow](kanalwechsel.south -| wlanverbindung.north)
		--(wlanverbindung.north);%
%
\end{pgfonlayer}%






	% Background
	\begin{pgfonlayer}{background}%
	%Ermittle Top Koordinate (Node mit oberster Kante)
	%und left Koordinate (Node mit linkster Kante)
		\path let \p1 = (absint.north),%
			\p2 = (netzwerk.west)%
			in node (topleft)[shape=coordinate,inner sep=0pt] at (\x2,\y1) {};%
	%Ermittle Bottom Koordinate (Node mit unterster Kante)
	%und right Koordinate (Node mit rechtester Kante)
		\path let \p3 = (ctrlkomm.south),%
			\p4 = (wlanverbindung.east)%
			in node (bottomright)[shape=coordinate,inner sep=0pt] at (\x4,\y3) {};%
	
	    \path[fill=slightblue]%
	    	($(topleft)+(-0.25,0.25)$)%
	    		rectangle%
	    	($(bottomright)+(0.25,-0.25)$);%
	\end{pgfonlayer}%





	\foreach \x in {1,...,1}{%
		\ifnodedefined{mat\x}{%
			\aeundefinethesenodes{mat\x}%
		}{}%
		\ifnodedefined{entry}{%
			\aeundefinethesenodes{entry}%
		}{}%
		\ifnodedefined{exit}{%
			\aeundefinethesenodes{exit}%
		}{}%
	}%
%
\normalfont%
%
\end{tikzpicture}%
%
%
\DenKrTikzStandalonePost%