\input{"./2ProjectSetup".tex}%
\input{"\DenKrLayoutMainRootDir/2includes/packages/preamble_pre".tex}%
\documentclass[tikz,fontsize=11pt,class=scrbook]{standalone}% I.e. the content from \input{"\DenKrLayoutMainRootDir/2layout/tikz_standalone/preamble_1_class".tex}%
\input{"\DenKrLayoutBaseRootDir/tikz_standalone/1TikzStandalonePicIncludeThis".tex}%
\DenKrTikzStandalonePre%
%
%
%
%
%
%
%
%
% \begin{document}%
\newcommand\nodedistance{1.0cm}%
\newcommand\innabst{0.8em}% Abstand zwischen Text und Zellenrand
%
\let\breite\undefined%
\newlength\breite%
\settowidth{\breite}{\large \textbf{Interfacename}}%
\setlength\breite{\breite+\innabst+\innabst+1em}%
%
% Define the layers to draw the diagram
% \pgfdeclarelayer{background}%
%\pgfdeclarelayer{foreground}
% \pgfsetlayers{background,main}%,foreground}
%
%
\begin{tikzpicture}[%
	scale = \tikzpicturescale,%
	auto,%
	node distance=\nodedistance%
]%
\newcommand{\cnt}{2}%
%
\definecolor{critred}{RGB}{255,70,40}%
\definecolor{entryred}{RGB}{255,50,50}%
\definecolor{slightblue}{RGB}{230,230,255}%
\definecolor{processblue}{RGB}{180,180,255}%
\definecolor{processblue2}{RGB}{90,180,250}%
%
\tikzstyle{background}=[%
	rectangle,%
	rounded corners,%
	minimum width=3cm,%
	minimum height=1cm,%
	text centered,%
	draw=black,%
	fill=slightblue%
]%
\tikzstyle{process1} = [%
	rectangle,%
	minimum width=3cm,%
	minimum height=1cm,%
	text centered,%
	align=center,%
	inner sep=\innabst,%
	draw=black,%
	fill=processblue%
]%
\tikzstyle{processsplit} = [%
	minimum width=3cm,%
	%minimum height=1cm,%
	text centered,%
	align=left,%
	inner sep=0.5*\innabst,%
	draw=black,%
	rectangle split,%
	rectangle split parts=#1,%
	rectangle split part fill={processblue,processblue2},%
    rectangle split every empty part={},% delete existing height, depth and width
	rectangle split empty part height=0.1ex,%
	rectangle split empty part depth=0ex,%
% 	shade,%
% 	shading=axis,%
% 	top color=white,%
% 	bottom color=processblue,%
% 	shading angle=45,%
]%
%
%
\begin{pgfonlayer}{main}%
% \bfseries%
%
%
\node(pfadEins)[inner sep=0pt,anchor=north west]at(0,0)
    {\includestandalone[mode=buildnew]{\DenKrTikzRootDir/example/ollerus-dateiordner-pfade-1}};
\node(pfadZwei)[inner sep=0pt,anchor=north west,xshift=\nodedistance]
	at(pfadEins.north east)
    {\includestandalone[mode=buildnew]{\DenKrTikzRootDir/example/ollerus-dateiordner-pfade-2}};
\node(pfadDrei)[inner sep=0pt,anchor=north west,xshift=\nodedistance]
	at(pfadZwei.north east)
    {\includestandalone[mode=buildnew]{\DenKrTikzRootDir/example/ollerus-dateiordner-pfade-3}};
%
\node(pfadEinsName)[inner sep=0pt,anchor=south west,align=left]
	at(pfadEins.north west)
	{Pfad \circlearound{1}:\\ /home/\$user\$/ollerus/};%
\node(pfadZweiName)[inner sep=0pt,anchor=south west,align=left]
	at(pfadZwei.north west)
	{Pfad \circlearound{2}:\\ /usr/root/ollerus/};%
\node(pfadDreiName)[inner sep=0pt,anchor=south west,align=left]
	at(pfadDrei.north west)
	{Pfad \circlearound{3}:\\ /etc/ollerus/};%
%
\end{pgfonlayer}%






	% Background
	\begin{pgfonlayer}{background}%
	%Ermittle Top Koordinate (Node mit oberster Kante)
	%und left Koordinate (Node mit linkster Kante)
% 		\path let \p1 = (optimizechannel.north),%
% 			\p2 = (scan.west)%
% 			in node (topleft)[shape=coordinate,inner sep=0pt] at (\x2,\y1) {};%
% 	%Ermittle Bottom Koordinate (Node mit unterster Kante)
% 	%und right Koordinate (Node mit rechtester Kante)
% 		\path let \p3 = (linkp4.south),%
% 			\p4 = (optimizechannel.east)%
% 			in node (bottomright)[shape=coordinate,inner sep=0pt] at (\x4,\y3) {};%
% 	
% 	    \path[fill=slightblue]%
% 	    	($(topleft)+(-0.25,0.25)$)%
% 	    		rectangle%
% 	    	($(bottomright)+(0.25,-0.25)$);%
	\end{pgfonlayer}%





%
\normalfont%
%
\end{tikzpicture}%
%
%
\DenKrTikzStandalonePost%