\input{"./2ProjectSetup".tex}%
\input{"\DenKrLayoutMainRootDir/2includes/packages/preamble_pre".tex}%
\documentclass[tikz,fontsize=11pt,class=scrbook]{standalone}% I.e. the content from \input{"\DenKrLayoutMainRootDir/2layout/tikz_standalone/preamble_1_class".tex}%
\input{"\DenKrLayoutBaseRootDir/tikz_standalone/1TikzStandalonePicIncludeThis".tex}%
\DenKrTikzStandalonePre%
%
%
%
\newcommand\nodedistance{0.5cm}%
\newcommand\innabst{0.8em}% Abstand zwischen Text und Zellenrand
%
\let\breite\undefined%
\newlength\breite%
\settowidth{\breite}{\large \textbf{Interfacename}}%
\setlength\breite{\breite+\innabst+\innabst+1em}%
%
\begin{tikzpicture}[%
	scale = \tikzpicturescale,%
	auto,%
	node distance=\nodedistance%
]%
\newcommand{\cnt}{2}%
%
%
% \bfseries%
%
%\draw[color=slightblue,fill] (-0.5,-4.5) rectangle (10.5,0.5);
%
	\node(threadwlanmon)%
		[note,anchor=north]%
		at(0,0)%
		{WLAN Monitoring Thread\\%
		Thread-ID: \ico{WLANMon\_ThID}};%
%
	\node(restartwlanmon_kommdest)%
		[cleartext,%
		left=3*\nodedistance of threadwlanmon]%
		{\textbf{\lipsaround{IT-wmon2}}};%
	\draw[arrowkomm,->](restartwlanmon_kommdest)--(threadwlanmon);%
%
	\node(startthread)%
		[startstop,%
		below=of threadwlanmon]%
		{Start};%
	\draw[arrow](threadwlanmon)--(startthread);%
% 	\node(threadwlanmonabove)[shape=coordinate,%
% 		above=0.5*\nodedistance of threadwlanmon]{};
% 	\draw[connect]($(threadstart.south)+(1cm,0)$)|-(threadwlanmonabove);%
% 	\draw[arrow](threadwlanmonabove)-|(threadwlanmon);%
%
	\node(killsignaloccur)%
		[interthread,anchor=north]%
		at(-16,0)%
		{Signal Ereignis\\
		\ico{SIGKILL\_WLANMON}};%
	\node(threadwlanmon)%
		[note,align=left,%
		below=of killsignaloccur]%
		{%
		\lstinline[language=C_var,basicstyle=\ttfamily]%
		{static void sigkill_wlanmon_handler()}\{\\%
		\ \ \ \ \lstinline[language=C_var,basicstyle=\ttfamily]%
		{pcap_breakloop(pcaphandle);}\\%
		\ \ \ \ \lstinline[language=C_var,basicstyle=\ttfamily]%
		{pcap_close(pcaphandle);}\\%
		\ \ \ \ \lstinline[language=C_var,basicstyle=\ttfamily]%
		{pthread_cancel(WLANMon_ThID);}\\%
		\ \ \ \ \lstinline[language=C_var,basicstyle=\ttfamily]%
		{pthread_join(WLANMon_ThID);}\\%
% 		\ \ \ \ \lstinline[language=C_var,basicstyle=\ttfamily]%
% 		{if(*wlanp)}\\%
% 		\ \ \ \ \ \ \ \ \lstinline[language=C_var,basicstyle=\ttfamily]%
% 		{free(*wlanp);}\\%
		\}%
		};%
	\draw[arrow](killsignaloccur)--(threadwlanmon);%
	\node(threadend)%
		[startstop,%
		below=of threadwlanmon]%
		{Ende};%
	\draw[arrow](threadwlanmon)--(threadend);%
%
	\node(stopwlanmon_kommdest)%
		[cleartext,%
		right=3*\nodedistance of killsignaloccur]%
		{\textbf{\lipsaround{IT-wmon1}}};%
	\draw[arrowkomm,->](stopwlanmon_kommdest)--(killsignaloccur);%
%
	\node(signal)%
		[process,align=right,%
		below=of startthread]%
		{\lstinline[language=C_var,basicstyle=\ttfamily]%
		{signal(SIGKILL_WLANMON,}\ \ \ \ \ \ \\%
		\lstinline[language=C_var,basicstyle=\ttfamily]%
		{sigkill_wlanmon_handler);}%
		};%
	\draw[arrow](startthread)--(signal);%
%
	\node(chanloopstart)%
		[process,%
		below=1*\nodedistance of signal]%
		{\ico{chan=1} (for-Loop)};%
	\draw[arrow](signal)--(chanloopstart);%
%
	\node(chanloopcond)%
		[decision,%
		below=1.5*\nodedistance of chanloopstart]%
		{chan <= 13?};%
	\draw[arrow](chanloopstart)--(chanloopcond);%
	\node(chanloopreconnect)[shape=coordinate]%
		at($(chanloopcond.north)+(-12cm,0.75*\nodedistance)$){};
	\draw[arrow](chanloopreconnect)--(chanloopreconnect-|chanloopcond.north);%
%
	\node(chanloopreset)%
		[process,%
		left=2.75*\nodedistance of chanloopcond]%
		{\ico{chan=1}};%
	\draw[arrow](chanloopcond)--%
		node[pos=0,anchor=south east]{Nein}%
		(chanloopreset);%
%
	\node(changeifconf)%
		[processsplitl=2,%
		xshift=-2.75cm,%
		below=2*\nodedistance of chanloopcond]%
		{\nodepart{one}WLAN Device IfType zu Monitoring ändern%
		\nodepart{two}WLAN Device Kanal auf Freq ändern};%
	\draw[arrow](chanloopcond)--%
		node[pos=0,anchor=north east]{Ja}%
		(chanloopcond|-changeifconf.north);%
	\draw[arrow](chanloopreset.south)--(chanloopreset.south|-changeifconf.north);%
%
	\node(devopen)%
		[process,%
		below=of changeifconf]%
		{WLAN Device zur Paket\\ Aufzeichnung "`öffnen"'};%
	\draw[arrow](changeifconf)--(devopen);%
%
	\node(datalinkswitch)%
		[mux,%
		text width=10.88cm,%
		xshift=-2cm,%
		below=of devopen]%
		{Link Typ des geöffnete Device};%
	\draw[arrow](devopen.south)--(datalinkswitch.north-|devopen.south);%
%
	\node(datalinkswitch3)%
		[muxwahl2,%
		anchor=north,%
		xshift=1.5cm]%
		at(datalinkswitch.south)%
		{};%
	\node(datalinkswitch3_text)%
		[cleartext,anchor=center,inner sep=1em]%
		at(datalinkswitch3)%
		{\textbf{\ldots}};%
	\node(datalinkswitch4)%
		[muxwahl2,%
		anchor=north,%
		xshift=4.5cm]%
		at(datalinkswitch.south)%
		{};%
	\node(datalinkswitch4_text)%
		[cleartext,anchor=north,inner sep=6pt]%
		at(datalinkswitch4.north)%
		{Radiotap};%
	\node(datalinkswitch2)%
		[muxwahl2,%
		anchor=north,%
		xshift=-1.5cm]%
		at(datalinkswitch.south)%
		{};%
	\node(datalinkswitch2_text)%
		[cleartext,anchor=north,inner sep=6pt]%
		at(datalinkswitch2.north)%
		{Prism};%
	\node(datalinkswitch1)%
		[muxwahl2,%
		anchor=north,%
		xshift=-4.5cm]%
		at(datalinkswitch.south)%
		{};%
	\node(datalinkswitch1_text)%
		[cleartext,anchor=north,inner sep=6pt]%
		at(datalinkswitch1.north)%
		{Ethernet};%
%
	\node(returnbaddevice)%
		[startstop,%
		below=of datalinkswitch1]%
		{Return\\ \ico{BAD\_DEVICE}};%
	\draw[arrow](datalinkswitch1)--(returnbaddevice);%
%
	\node(starttime)%
		[process,%
		xshift=-1cm,%
		below=of datalinkswitch4]%
		{Erfasse \ico{starttime}};%
	\draw[arrow](datalinkswitch4.south)--(starttime.north-|datalinkswitch4.south);%
%
	\node(aktuellezeit)%
		[processsplit=2,%
		below=1.5*\nodedistance of starttime]%
		{\nodepart{one}Erfasse \ico{systime}%
		\nodepart{two}Berechne\\ \ico{difftime(systime,starttime)}};%
	\draw[arrow](starttime)--(aktuellezeit);%
	\node(timeloopstart)[shape=coordinate]%
		at($(aktuellezeit.north west)+(-2cm,0.75*\nodedistance)$){};%
	\draw[arrow](timeloopstart)--(timeloopstart-|aktuellezeit);%
%
	\node(difftime)%
		[decision,%
		below=of aktuellezeit]%
		{\ico{difftime >}\\ \ico{TimeToMonitor}?};%
	\draw[arrow](aktuellezeit)--(difftime);%
%
	\node(grabpacket)%
		[processsplit=2,%
		below=1.5*\nodedistance of difftime]%
		{\nodepart{one}Paket entnehmen%
		\nodepart{two}\ico{packetcounter++;}};%
	\draw[arrow](difftime)--%
		node[pos=0,anchor=north west]{Nein}%
		(grabpacket);%
%
	\node(decodeheader)%
		[processsplit=2,%
		below=1*\nodedistance of grabpacket]%
		{\nodepart{one}Dekodiere Radiotap Header%
		\nodepart{two}Dekodiere IEEE 802.11 Header};%
	\draw[arrow](grabpacket)--(decodeheader);%
%
	\node(checkmacloopstart)%
		[process,%
		below=1*\nodedistance of decodeheader]%
		{\ico{i=0} (for-Loop)};%
	\draw[arrow](decodeheader)--(checkmacloopstart);%
%
	\node(macexistsloopcond)%
		[decision,%
		below=1.5*\nodedistance of checkmacloopstart]%
		{\ico{i <}\\%
		\ico{(*wlanp)->count}};%
	\draw[arrow](checkmacloopstart)--(macexistsloopcond);%
	\node(checkmacloopstartreconnect)[shape=coordinate]%
		at($(macexistsloopcond.north)+(-4cm,0.75*\nodedistance)$){};%
	\draw[arrow](checkmacloopstartreconnect)--(macexistsloopcond.north|-checkmacloopstartreconnect);%
%
	\node(macexists)%
		[decision,%
		below=1.25*\nodedistance of macexistsloopcond]%
		{MAC bereits\\ erfasst?};%
	\draw[arrow](macexistsloopcond)--%
		node[pos=0,anchor=north east]{Ja}%
		(macexists);%
	\draw[arrow](macexists)--%
		node[pos=0,anchor=south east]{Ja}%
		(macexists.west-|timeloopstart);%
	\node(macexistsloopcondfinish)[shape=coordinate,xshift=0.75*\nodedistance]%
		at(macexistsloopcond.east){};%
	\draw[connect](macexistsloopcond)--%
		node(rightmostnein)[pos=0,anchor=south west]{Nein}%
		(macexistsloopcondfinish);%
%
	\node(checkmacloopend)%
		[process,%
		below=1.25*\nodedistance of macexists]%
		{\ico{i++}};%
	\draw[arrow](macexists)--%
		node[pos=0,anchor=north east]{Nein}%
		(checkmacloopend);%
	\draw[connect](checkmacloopend)-|(checkmacloopstartreconnect);%
%
	\node(createdata)%
		[processsplitl=3,%
		rectangle split part align={center, left, left},%
		below=1.5*\nodedistance of checkmacloopend]%
		{\nodepart{one}Datensatz erstellen%
		\nodepart{two}Quell-MAC Adresse dem standard-\\ konformen WLAN-Header
		entnehmen%
		\nodepart{three}RSSI dem Radiotap-Header entnehmen};%
	\draw[connect](macexistsloopcondfinish)|-($(createdata.north)+(0,0.75*\nodedistance)$);%
	\draw[arrow]($(createdata.north)+(0,0.75*\nodedistance)$)--(createdata.north);%
	\draw[connect](createdata.west)-|(timeloopstart);%
%
	\node(devclose)%
		[process,%
		below=1.5*\nodedistance of createdata]%
		{WLAN Device von Paket\\ Aufzeichnung "`schließen"'};%
	\draw[arrow]($(devclose.north)+(0,0.75*\nodedistance)$)--(devclose);%
%
	\node(leavepcaploop)[shape=coordinate,xshift=0.75*\nodedistance]%
		at(rightmostnein.east|-difftime.east){};%
	\draw[connect](difftime.east)--(leavepcaploop);%
	\draw[connect](leavepcaploop)|-($(devclose.north)+(0,0.75*\nodedistance)$);%
%
	\node(sniffend)%
		[startstop,%
		left=1.5*\nodedistance of devclose]%
		{Funktion\\ Ende};%
	\draw[arrow](devclose)--(sniffend);%
%
	\node(putinshmem)%
		[interthread,%
		below=1.5*\nodedistance of sniffend]%
		{ShMem Zugriff};%
	\draw[arrow](sniffend)--(putinshmem);%
%
	\node(fillshmem)%
		[processsplitl=3,%
		align=right,%
		rectangle split part align={right, center, center},%
		below=1*\nodedistance of putinshmem]%
		{\nodepart{one}Set ShMem Type \ico{AIC\_AITC\_WLANSTAT\_COMPLETE}\\
			oder \ico{AIC\_AITC\_WLANSTAT\_TRAFFICSTAT}%
		\nodepart{two}Set ShMem Flags \ico{NEW | SEND}%
		\nodepart{three}Daten konkateniert ins ShMem schreiben.};%
	\draw[arrow](putinshmem)--(fillshmem);%
%
	\node(semsendpostswitch)%
		[process,%
		below=1*\nodedistance of fillshmem]%
		{\lstinline[language=C_var,basicstyle=\ttfamily]%
		{sem_post(&sem_send);}};%
	\draw[arrow](fillshmem)--(semsendpostswitch);%
%
	\node(sendtoctrl)%
		[interthread,%
		below=1*\nodedistance of semsendpostswitch]%
		{ShMem Freigabe};%
	\draw[arrow](semsendpostswitch)--(sendtoctrl);%
%
	\node(chanloopend)%
		[process,%
		below=1*\nodedistance of sendtoctrl]%
		{\ico{chan += 4}};%
	\draw[arrow](sendtoctrl)--(chanloopend);%
	\draw[connect](chanloopend.west)-|(chanloopreconnect);%
%
%
%
%
%
%
%
%
%
%
%
%
%	Border around the Sniffing Function 
		\path let \p1 = (changeifconf.north),%
			\p2 = (datalinkswitch.west)%
			in node (topleftnot)[shape=coordinate,inner sep=0pt] at (\x2,\y1) {};%
		\path let \p3 = (sniffend.south),%
			\p4 = (rightmostnein.east)%
			in node (bottomrightnot)[shape=coordinate,inner sep=0pt] at (\x4,\y3) {};%
%
		\node(topleft)[shape=coordinate,inner sep=0pt]
			at($(topleftnot)+(-0.75*\nodedistance,0.75*\nodedistance)$){};
		\node(bottomright)[shape=coordinate,inner sep=0pt]
			at($(bottomrightnot)+(0.75*\nodedistance,-0.75*\nodedistance)+(0.3cm,0)$){};
		\node(bottomleft)[shape=coordinate,inner sep=0pt]at(topleft|-bottomright){};%
		\node(topright)[shape=coordinate,inner sep=0pt]at(topleft-|bottomright){};%
%
	    \path[draw=black,loosely dotted]%
	    	(topleft)%
	    		rectangle%
	    	(bottomright);%
%
	    \node[cleartext,rotate=-90,anchor=south west,inner sep=1ex]%
	    	at(topleft-|bottomright)%
	    	{\lstinline[language=C_var,basicstyle=\ttfamily]%
			{wifi_package_parse(dev, freq, **wlanp)}};%
%
%
%
%
%
%
%
\normalfont%
%
\end{tikzpicture}%
%
%
\DenKrTikzStandalonePost%