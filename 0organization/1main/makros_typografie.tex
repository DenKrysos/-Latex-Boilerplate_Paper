%####################################################################
%====================================================================
%------ When using this, include it after "makros.tex" --------------
%vvvvvvvvvvvvvvvvvvvvvvvvvvvvvvvvvvvvvvvvvvvvvvvvvvvvvvvvvvvvvvvvvvvv
%
%
%
%
%####################################################################
%====================================================================
%------------- Framed TextBox with Banner-Header -------------
%vvvvvvvvvvvvvvvvvvvvvvvvvvvvvvvvvvvvvvvvvvvvvvvvvvvvvvvvvvvvvvvvvvvv
%____________________________________________________________________
%Args: width, HeaderName, BannerColor, BorderColor, BG-Color
\newenvironment{TBoxColored}[6][]{%
    \colorlet{colorBoxBanner}{#4}%
    \colorlet{colorBoxBorder}{#5}%
    \colorlet{colorBoxBG}{#6}%
    \noindent\begin{tikzpicture}[#1]%
        \def\boxHead{#3}%
        \node[draw=colorBoxBorder,fill=colorBoxBG,inner sep=1.5ex,rounded corners=0.5em,text width=#2]% good options: minimum height, minimum width
            (BOXCONTENT)\bgroup\ignorespaces%%(BOXCONTENT)\bgroup\rule{0pt}{3ex}\ignorespaces%
}%
%---------------------------------------------
{%
        \egroup;%
        \node[right,inner xsep=1em,fill=colorBoxBanner!75,outer sep=0pt,text height=2ex,text depth=.5ex] (BOXNAME)%
            at ([shift={(-1em,0pt)}]BOXCONTENT.north west){\boxHead};%
        \fill[colorBoxBanner](BOXNAME.north east)--+(-1em,1em)--+(-1em,0)--cycle;%
        \fill[colorBoxBanner](BOXNAME.south west)--+(1em,-1em)--+(1em,0)--cycle;%
    \end{tikzpicture}%
}%
%^^^^^^^^^^^^^^^^^^^^^^^^^^^^^^^^^^^^^^^^^^^^^^^^^^^^^^^^^^^^^^^^^^^^
%------------- Framed TextBox with Banner-Header End -------------
%====================================================================
%####################################################################
%
%
%
%####################################################################
%====================================================================
%------------- Framed TextBox with Banner-Header, DoubleBorder -------------
%vvvvvvvvvvvvvvvvvvvvvvvvvvvvvvvvvvvvvvvvvvvvvvvvvvvvvvvvvvvvvvvvvvvv
%____________________________________________________________________
\let\DoubleBorderDist\undefined%
\newlength\DoubleBorderDist%
\setlength\DoubleBorderDist{.75ex}% Abstand zwischen Text und Zellenrand
%
\let\ingredientsTextSep\undefined%
\newlength\ingredientsTextSep%
\setlength\ingredientsTextSep{1.5ex}% Abstand zwischen Text und Zellenrand
%
%Args: width, HeaderName, BannerColor, BorderColor, BG-Color, BorderWidth
\newenvironment{TBoxColoredDB}[7][]{%
    \colorlet{colorBoxBanner}{#4}%
    \colorlet{colorBoxBorder}{#5}%
    \colorlet{colorBoxBG}{#6}%
    \noindent\begin{tikzpicture}[#1]%
        \def\boxHead{#3}%
        \node[draw=colorBoxBorder,line width=#7,inner sep=\DoubleBorderDist,rounded corners=0.5em]%,text width=#2]% good options: minimum height, minimum width
            (BOXCONTENT)\bgroup\ignorespaces%%(BOXCONTENT)\bgroup\rule{0pt}{3ex}\ignorespaces%
        \noindent\begin{tikzpicture}[#1]%
            \node[fill=colorBoxBG,inner sep=\ingredientsTextSep,rounded corners=0.5em,text width=#2]% good options: minimum height, minimum width
                (BOXCONTENT_INNER)\bgroup\ignorespaces%%(BOXCONTENT_INNER)\bgroup\rule{0pt}{3ex}\ignorespaces%
}%
%---------------------------------------------
{%
            \egroup;%
        \end{tikzpicture}%
        \egroup;%
        \node [right,inner xsep=1em,fill=colorBoxBanner!75,outer sep=0pt,text height=2ex,text depth=.5ex] (BOXNAME) %
            at ([shift={(-1em,0pt)}]BOXCONTENT.north west) {\boxHead};%
        \fill[colorBoxBanner] (BOXNAME.north east) -- +(-1em,1em) -- +(-1em,0) -- cycle;%
        \fill[colorBoxBanner] (BOXNAME.south west) -- +(1em,-1em) -- +(1em,0) -- cycle;%
    \end{tikzpicture}%
}%
%^^^^^^^^^^^^^^^^^^^^^^^^^^^^^^^^^^^^^^^^^^^^^^^^^^^^^^^^^^^^^^^^^^^^
%------------- Framed TextBox with Banner-Header, DoubleBorder End -------------
%====================================================================
%####################################################################
%
%
%
%####################################################################
%====================================================================
%------------- Rezept Zutaten Box -------------
%vvvvvvvvvvvvvvvvvvvvvvvvvvvvvvvvvvvvvvvvvvvvvvvvvvvvvvvvvvvvvvvvvvvv
%____________________________________________________________________
\xdefinecolor{colorIngredientsHead}{RGB}{62,96,111}% Kind-of a Dark Neutral Blue
\definecolor{colorIngredientsBorder}{RGB}{50,0,110}
\definecolor{colorIngredientsBG}{RGB}{255,230,180}
\colorlet{colorDishPicBG}{colorIngredientsBG!40}
%
\let\recipeBorderWidth\undefined%
\newlength\recipeBorderWidth%
\setlength\recipeBorderWidth{0.75pt}%
\let\recipeBorderWidthIngredients\undefined%
\newlength\recipeBorderWidthIngredients%
\setlength\recipeBorderWidthIngredients{\recipeBorderWidth}%
\let\recipeBorderWidthDishPic\undefined%
\newlength\recipeBorderWidthDishPic%
\setlength\recipeBorderWidthDishPic{\recipeBorderWidth}%
%
\let\ingredientsWidthHaveToReduce\undefined%
\newlength\ingredientsWidthHaveToReduce%
\setlength\ingredientsWidthHaveToReduce{2\DoubleBorderDist+\recipeBorderWidth+2\ingredientsTextSep+1em}%2\DoubleBorderDist+2\recipeBorderWidth+2\ingredientsTextSep+'HowMuchTheHeadingFlagJutsOut:1em'
\let\ingredientsMaxWidth\undefined%
\newlength\ingredientsMaxWidth%
\setlength\ingredientsMaxWidth{\textwidth-\ingredientsWidthHaveToReduce}%
%
\newenvironment{ingredients_base}[2][]{%
    \begin{TBoxColoredDB}{#2}{\textit{\underline{Zutaten}}}{colorIngredientsHead}{colorIngredientsBorder}{colorIngredientsBG}{\recipeBorderWidthIngredients}%
    \vspace*{-\baselineskip}%
    \setlength{\columnseprule}{0.4pt}%
    \begin{multicols}{2}%
}%
%---------------------------------------------
{%
    \end{multicols}%
    \setlength{\columnseprule}{0pt}%
    \end{TBoxColoredDB}%
}%
%===========================================
\newenvironment{ingredients}[2][]{%This wraps the actual formatting of the Box into the checking of the Draft-State Value
    \ifnumcomp{\value{draftPhase_ingredients}}{=}{0}{%
        \begin{ingredients_base}{#2}%
    }{%
        \ifnumcomp{\value{draftPhase_ingredients}}{=}{1}{%
            \textit{\underline{Zutaten}}:\nl%
        }{%
            »Invalid Value on \enquote{draftPhase} / \enquote{draftPhase\_ingredients}«%
        }%
    }%
}%
%---------------------------------------------
{%
    \ifnumcomp{\value{draftPhase_ingredients}}{=}{0}{%
        \end{ingredients_base}%
    }{%
        \ifnumcomp{\value{draftPhase_ingredients}}{=}{1}{%
        }{%
            »Invalid Value on \enquote{draftPhase} / \enquote{draftPhase\_ingredients}«%
        }%
    }%
}%
%^^^^^^^^^^^^^^^^^^^^^^^^^^^^^^^^^^^^^^^^^^^^^^^^^^^^^^^^^^^^^^^^^^^^
%------------- Rezept Zutaten Box End -------------
%====================================================================
%####################################################################
%
%
%
%
%####################################################################
%====================================================================
%------------- Picture of the Dish -------------
%vvvvvvvvvvvvvvvvvvvvvvvvvvvvvvvvvvvvvvvvvvvvvvvvvvvvvvvvvvvvvvvvvvvv
%____________________________________________________________________
\newcommand{\dishPicSingle}[1]{%
\begin{center}%\begin{scope}%\begin{figure}[!htbp]%[H]
\centering%
    \begin{tikzpicture}%
        \colorlet{colorBoxBorder}{colorIngredientsBorder}%
        \colorlet{colorBoxBG}{colorDishPicBG}%
        \node[draw=colorBoxBorder,fill=colorBoxBG,inner sep=1.5ex,rounded corners=0.5em]%
            (BOXCONTENT)\bgroup\ignorespaces%%(BOXCONTENT)\bgroup\rule{0pt}{3ex}\ignorespaces%
            	\includegraphics[width=0.47\textwidth]{#1}%
            \egroup;%
    \end{tikzpicture}%
	%\captionof{figure}{text}\label{labelname}%%\caption{Curry-Pfanne}%%\label{fig:curry-pfanne}%
\end{center}%\end{scope}%\end{figure}%
}%
%====================================================================
\newcommand{\dishPicDouble}[2]{%
\begin{center}%\begin{scope}%\begin{figure}[!htbp]%[H]
\centering%
    \begin{tikzpicture}%
        \colorlet{colorBoxBorder}{colorIngredientsBorder}%
        \colorlet{colorBoxBG}{colorDishPicBG}%
        \node[draw=colorBoxBorder,fill=colorBoxBG,inner sep=1.5ex,rounded corners=0.5em,rectangle split,rectangle split horizontal,rectangle split parts=2]%
            (BOXCONTENT)\bgroup\ignorespaces%%(BOXCONTENT)\bgroup\rule{0pt}{3ex}\ignorespaces%
            	\includegraphics[width=0.47\textwidth]{#1}%
            	\nodepart{two}%
            	\includegraphics[width=0.47\textwidth]{#2}%
            \egroup;%
            %\draw[brown](current bounding box.south west)rectangle(current bounding box.north east);
    \end{tikzpicture}%
	%\captionof{figure}{text}\label{labelname}%%\caption{Curry-Pfanne}%%\label{fig:curry-pfanne}%
\end{center}%\end{scope}%\end{figure}%
}%
%====================================================================
%TODO: Define a maximum height. Then check, if including a picture with the width exceeds this max height. In case: rather use the height.
%TODO: Check if a picture is too small. Means: using a width to include it would stretch it. Rather use the actual width of the picture in that case.
%TODO: The Value 2 for the Stretch-Option
%ARGS: [Optional] Number of Pictures per Line. Defaults to 2
%       [Optional] Picture-Surpressing Override. Displays this Picture-Matrix no matter if the global Value for surpressing pictures is set. Obviously, to use this, the first Optional Argument has to be set -> so just add it and pass it the default-value of '2', even if you actually not intend to use it.
%       A List of all pictures with by path, inside one {}, divided with |. E.g.: {graphics/Pic1.png | graphics/Pic2.jpg}
%       [Optional] Stretching Option. Stretches the last line if the number of remaining pictures does not equal the numer of pictures per Line. Valid Values for Stretch:
%       1: Only stretches the frame, picture inside stays same size as others
%       2: Also stretches the picture to fit either its maximum width or its width fits the page width. % TODO, not done yet
% Usage-Example: \dishPicMat[3]{graphics/Pic1.png | graphics/Pic2.jpg}
%       Example2: \dishPicMat{graphics/Pic1.png}
%       Example3(Stretch): \dishPicMat[2]{graphics/Pic1.png | graphics/Pic2.jpg | graphics/Pic3.jpg}[1]
\DeclareDocumentCommand{\dishPicMat}{%
    O{2} O{0} >{\SplitList{|}}m O{0}%
}{%
    %\newcounter{quotient}%
    \setcounter{numArgs}{0}%
    \setcounter{numPerLine}{\number\numexpr#1\relax}%
    \setcounter{ArrayEntCnt}{0}%
    \ProcessList{#3}{\preparePictureMatrixList}%
    \setcounter{numLines}{\value{numArgs}}%
    \divide\value{numLines} by \value{numPerLine}%
    %\setcounter{numLines}{\inteval{\value{numArgs}/\value{numPerLine}}}%
    %\setcounter{numInLastLine}{\value{numArgs}-(\value{numLines}-1)*\value{numPerLine}}%
    \setcounter{numInLastLine}{\value{numArgs}-\value{numLines}*\value{numPerLine}}%
    \ifnumcomp{\value{numInLastLine}}{>}{0}{%
        \setcounter{numLines}{\value{numLines}+1}%
    }{%
        \setcounter{numInLastLine}{\value{numPerLine}}%
    }%
    %\the\value{numArgs} -- \the\value{numPerLine} -- \the\value{numLines} -- \the\value{numInLastLine}
    \let\innAbst\undefined%
    \newlength\innAbst%
    \setlength\innAbst{1.5ex}% Abstand zwischen Text und Zellenrand
    \let\lineHeight\undefined%
    \newlength\lineHeight%
    \setlength\lineHeight{0pt}%
    \let\lineHeightCmp\undefined%
    \newlength\lineHeightCmp%
    \setlength\lineHeightCmp{0pt}%
    \let\nodeWidth\undefined%
    \newlength\nodeWidth%
    \setlength\nodeWidth{\textwidth/\value{numPerLine}-\recipeBorderWidthDishPic-\recipeBorderWidthDishPic/\value{numPerLine}}%
    % \ifnumcomp{\thenumArgs}{<}{\thenumPerLine}{%
    %     \setlength\nodeWidth{\textwidth/\value{numArgs}-\recipeBorderWidthDishPic-\recipeBorderWidthDishPic/\value{numArgs}}%
    % }{%
    %     \setlength\nodeWidth{\textwidth/\value{numPerLine}-\recipeBorderWidthDishPic-\recipeBorderWidthDishPic/\value{numPerLine}}%
    % }%
    \let\picWidth\undefined%
    \newlength\picWidth%
    \setlength\picWidth{\nodeWidth-2\innAbst}% Abstand zwischen Text und Zellenrand
    %\setlength\picWidth{\textwidth/\value{numPerLine}-2\innAbst}% Abstand zwischen Text und Zellenrand
    %\the\picWidth -- \the\textwidth
    \setcounter{prevNode}{0}%
    \setcounter{currLine}{1}%
    \setcounter{currRow}{1}%
    % - - - - - - - - - - - - -
    \begin{center}
    \centering%
        \begin{tikzpicture}%
            % \usetikzlibrary{shapes.misc, positioning}%
            % \usetikzlibrary{calc}%
            % \pgfdeclarelayer{background}%
            % \pgfdeclarelayer{nodeShape}%->Layer1
            % \pgfdeclarelayer{nodePic}%->Layer2
            % \pgfsetlayers{background,nodeShape,nodePic}%
            \colorlet{colorBoxBorder}{colorIngredientsBorder}%
            \colorlet{colorBoxBG}{colorDishPicBG}%
            \tikzstyle{rectBase}=[inner sep=\innAbst,rectangle,line width=\recipeBorderWidthDishPic,anchor=north west, minimum width=\nodeWidth, minimum height=\lineHeight]%draw=colorBoxBorder,fill=colorBoxBG
            \tikzstyle{pathLineFill}=[draw=colorBoxBorder,fill=colorBoxBG,line width=\recipeBorderWidthDishPic]%
            \tikzstyle{cornRounding}=[rounded corners=0.5em]%
            \tikzstyle{cornNoRound}=[]%
            \coordinate(-1)at(0,0);%As first anchor, which can be referenced by the first actual node with id (0)
            % \begin{pgfonlayer}{Layer2}%
            % \node[rectBase](0)%
            %     \bgroup\ignorespaces%
            %     	\includegraphics[width=\picWidth]{\checkArrayEnt{0}{NoEntryDefined}{\getArrayEnt{0}}}%
            %     \egroup;%
            % \end{pgfonlayer}%
            % \drawPicNodeRect{0}{cornRounding}{cornNoRound}{cornNoRound}{cornNoRound}
            \setcounter{loopi}{0}%
            \loop\ifnum\value{loopi}<\value{numArgs}%
                \ifnumcomp{\thecurrRow}{>}{\thenumPerLine}{%
                    \stepcounter{currLine}%
                    \setcounter{currRow}{1}%
                    \setcounter{prevNode}{\theloopi-\thenumPerLine}%
                    \renewcommand{\newNodeRefPoint}{south west}%
                }{%
                    \setcounter{prevNode}{\theloopi-1}%
                    \renewcommand{\newNodeRefPoint}{north east}%
                }%
                \ifnumcomp{\thecurrRow}{=}{1}{%
                    \setcounter{loopj}{\theloopi}%
                    \ifnumcomp{\thecurrLine}{=}{\thenumLines}{%
                        \setcounter{temp}{\theloopj+\thenumInLastLine}%
                    }{%
                        \setcounter{temp}{\theloopj+\thenumPerLine}%
                    }%
                        \setlength\lineHeight{1pt}
                    {\loop\ifnum\value{loopj}<\value{temp}%Inner loops must be embraced
                        \settoheight{\lineHeightCmp}{\includegraphics[width=\picWidth]{\checkArrayEnt{\theloopj}{}{\getArrayEnt{\theloopj}}}}\global\lineHeightCmp=\lineHeightCmp%
                        \setlength\lineHeightCmp{\lineHeightCmp+2\innAbst}\global\lineHeightCmp=\lineHeightCmp%
                        \ifnumcomp{\lineHeightCmp}{>}{\lineHeight}{%
                            \setlength\lineHeight{\lineHeightCmp}\global\lineHeight=\lineHeight%
                        }{}%
                    \stepcounter{loopj}%
                    \repeat}%
                }{}%
                \ifthenelse{\number#4=1 \AND \thenumInLastLine<\thenumPerLine}{%
                    \ifnumcomp{\thecurrLine}{=}{\thenumLines}{%
                        %\setlength\nodeWidth{\textwidth/\value{numPerLine}-\recipeBorderWidthDishPic-\recipeBorderWidthDishPic/\value{numPerLine}}%
                        \ifnumcomp{\thecurrRow}{=}{1}{%
                            \setlength\nodeWidth{\textwidth/\value{numInLastLine}-\recipeBorderWidthDishPic-\recipeBorderWidthDishPic/\value{numInLastLine}}%
                        }{}%
                        \ifnumcomp{\thecurrRow}{=}{\thenumInLastLine}{%
                            %\setlength\nodeWidth{\nodeWidth+\recipeBorderWidthDishPic}%
                        }{}%
                    }{}%
                }{}%
                \begin{pgfonlayer}{Layer2}%
                    \node[rectBase](\theloopi)%
                        at (\theprevNode.\newNodeRefPoint)%
                        \bgroup\ignorespaces%
                            %\theloopi -- \thecurrLine%
                        	%\the\lineHeight - \the\value{loopj}, \the\value{temp}
                            \ifnumcomp{\value{draftPhase_pic}}{=}{0}{%
                                \dishPicMatIncludegraphics%
                            }{%
                                \ifnumcomp{\value{draftPhase_pic}}{=}{1}{%
                                    \ifnumcomp{#2}{=}{1}{%
                                        \dishPicMatIncludegraphics%
                                    }{%
                                        \textbf{»Draft«}%
                                    }%
                                }{%
                                    »Invalid Value on \enquote{draftPhase} / \enquote{draftPhase\_pic}«%
                                }%
                            }%
                        \egroup;%
                \end{pgfonlayer}%
                \ifthenelse{\thecurrLine=1 \AND \thecurrRow=1}{%
                    \renewcommand{\cornFormatLT}{cornRounding}%
                }{}%
                \ifthenelse{\thecurrLine=1 \AND \thecurrRow=\thenumPerLine}{%
                    \renewcommand{\cornFormatRT}{cornRounding}%
                }{}%
                \ifthenelse{\thecurrLine=1 \AND \thecurrRow=\thenumArgs}{%
                    \renewcommand{\cornFormatRT}{cornRounding}%
                }{}%
                \ifthenelse{\thecurrLine=\thenumLines \AND \thecurrRow=1}{%
                    \renewcommand{\cornFormatLB}{cornRounding}%
                }{}%
                \ifthenelse{\thecurrLine=\thenumLines \AND \thecurrRow=\thenumInLastLine}{%
                    \renewcommand{\cornFormatRB}{cornRounding}%
                }{}%
                \ifthenelse{\number\numexpr#4\relax=1}{}{%
                    \setcounter{temp}{\thenumLines-1}%
                    \ifthenelse{\thenumInLastLine<\thenumPerLine \AND \thecurrLine=\thetemp \AND \thecurrRow=\thenumPerLine}{%
                        \renewcommand{\cornFormatRB}{cornRounding}%
                    }{}%
                }%
                \drawPicNodeRect{\theloopi}{\cornFormatLT}{\cornFormatRT}{\cornFormatRB}{\cornFormatLB}%
                \renewcommand{\cornFormatLT}{cornNoRound}%
                \renewcommand{\cornFormatRT}{cornNoRound}%
                \renewcommand{\cornFormatLB}{cornNoRound}%
                \renewcommand{\cornFormatRB}{cornNoRound}%
            \stepcounter{currRow}%
            \stepcounter{loopi}%
            \repeat%
        \end{tikzpicture}%
    \end{center}
}%
\newcommand{\dishPicMatIncludegraphics}{\includegraphics[width=\picWidth]{\checkArrayEnt{\theloopi}{NoEntryDefined}{\getArrayEnt{\theloopi}}}}%
%====================================================================
%####################################################################
%\newcommand\countListArgs[1]{\stepcounter{numArgs}}%
\newcommand\preparePictureMatrixList[1]{%
    \stepcounter{numArgs}%
    \addArrayEnt{#1}%
}%
%\dishPicMat neccessities
\newcounter{temp}%
\newcounter{numArgs}%
\newcounter{numPerLine}%
\newcounter{numLines}%
\newcounter{numInLastLine}%
\newcounter{prevNode}%
\newcounter{currLine}%
\newcounter{currRow}%
\newcounter{loopi}%
\newcounter{loopj}%
% = = = = = = = = = = = = = = = =
%Args: Either cornRounding or cornNoRound
%Arg-Order: Node-ID, NorthWest, NorthEast, SouthEast, SouthWest
\newcommand{\drawPicNodeRect}[5]{%
\begin{pgfonlayer}{Layer1}%
    \path[pathLineFill]($(#1.north)+(0.1pt,0)$){[#3]-|(#1.east)}{[#4]|-(#1.south)}{[#5]-|(#1.west)}{[#2]|-($(#1.north)+(-0.1pt,0)$)}--($(#1.north)+(0.1pt,0)$)--cycle;%
\end{pgfonlayer}%
}%
\newcommand{\cornFormatLT}{cornNoRound}%Corner:LeftTop
\newcommand{\cornFormatRT}{cornNoRound}%
\newcommand{\cornFormatLB}{cornNoRound}%
\newcommand{\cornFormatRB}{cornNoRound}%
% - - - - - - - - - - - - -
\newcommand{\newNodeRefPoint}{north}%
% - - - - - - - - - - - - -
%^^^^^^^^^^^^^^^^^^^^^^^^^^^^^^^^^^^^^^^^^^^^^^^^^^^^^^^^^^^^^^^^^^^^
%------------- Dish Pic End -------------
%====================================================================
%####################################################################
%
%
%
%
%
%
%####################################################################
%====================================================================
%------------- Some Array-alike Fun -------------
%vvvvvvvvvvvvvvvvvvvvvvvvvvvvvvvvvvvvvvvvvvvvvvvvvvvvvvvvvvvvvvvvvvvv
%____________________________________________________________________
\newcounter{ArrayEntCnt}%
\newcommand\addArrayEnt[1]{%
  \csdef{ArrayEnt\theArrayEntCnt}{#1}%
  \stepcounter{ArrayEntCnt}%
}%
\newcommand\getArrayEnt[1]{%
  \csuse{ArrayEnt#1}}%
\newcommand\checkArrayEnt[1]{%
  \ifcsempty{ArrayEnt#1}
}
%^^^^^^^^^^^^^^^^^^^^^^^^^^^^^^^^^^^^^^^^^^^^^^^^^^^^^^^^^^^^^^^^^^^^
%------------- Some Array-alike Fun -------------
%====================================================================
%####################################################################
%
%
%
%
%
%####################################################################
%====================================================================
%------------- Customizable Rating - the most versatile Macro. -------------
%------------- You have to put in the drawing of the actual node -------------
%------------- (see \ratingCustomized for an example) -------------
%vvvvvvvvvvvvvvvvvvvvvvvvvvvvvvvvvvvvvvvvvvvvvvvvvvvvvvvvvvvvvvvvvvvv
%____________________________________________________________________
% Think about: The Label of the drawn nodes serves as text baseline-anchor. So if you want the inline graphic to be totally visually vertically centered, you have to add a \raisebox to the text, in case you are using a different size than \normalfont
\let\baselineRaiseLARGE\undefined%
\newlength\baselineRaiseLARGE%
\setlength\baselineRaiseLARGE{-0.4ex}%
\let\baselineRaisefootnotesize\undefined%
\newlength\baselineRaisefootnotesize%
\setlength\baselineRaisefootnotesize{0.15ex}%
%- - - - - - - -
%ARGS: RatingPoints, RatingOutOf, InverseRating, FilledColor, EmptyColor, filledTextCol, emptyTextCol, NodeToDrawFilled, NodeToDrawEmpty
%     In case the InverseRating is given with 1 instead of 0 it is considered as "less points is better", which is then reflected in the coloring of the Rating-Ellipse
\newcommand\ratingCustomizedNeedNode[9]{%
    \colorlet{filledCol}{#4}%
    \colorlet{emptyCol}{#5}%
    \colorlet{filledTextCol}{#6}%
    \colorlet{emptyTextCol}{#7}%
    \colorlet{ratingColUsed}{emptyTextCol}%
    \definecolor{ratingColMax}{RGB}{127,0,255}%
    \definecolor{ratingColHigh}{RGB}{0,153,0}%
    \definecolor{ratingColMed}{RGB}{200,180,0}%
    \definecolor{ratingColLow}{RGB}{220,100,0}%{200,100,0}%{250,140,0}
    \definecolor{ratingColMin}{RGB}{210,0,0}%
    \setcounter{ArrayEntCnt}{0}%
    \addArrayEnt{ratingColMin}%
    \addArrayEnt{ratingColLow}%
    \addArrayEnt{ratingColMed}%
    \addArrayEnt{ratingColHigh}%
    \addArrayEnt{ratingColMax}%
    \def\ratingRatio{\numexpr#1*100/#2\relax}%
    \ifnumless{\ratingRatio}{91}{%
        \ifnumless{\ratingRatio}{65}{%
            \ifnumless{\ratingRatio}{35}{%
                \ifnumless{\ratingRatio}{11}{%
                    \setcounter{ratingCustomizedResultLevel}{0}%
                }{%
                    \setcounter{ratingCustomizedResultLevel}{1}%
                }%
            }{%
                \setcounter{ratingCustomizedResultLevel}{2}%
            }%
        }{%
            \setcounter{ratingCustomizedResultLevel}{3}%
        }%
    }{%
        \setcounter{ratingCustomizedResultLevel}{4}%
    }%
    \ifnumcomp{\numexpr#3\relax}{=}{1}{\setcounter{ratingCustomizedResultLevel}{4-\value{ratingCustomizedResultLevel}}}{}%
    \colorlet{ratingColUsed}{\checkArrayEnt{\theratingCustomizedResultLevel}{black}{\getArrayEnt{\theratingCustomizedResultLevel}}}%
    \let\innabst\undefined%
    \newlength\innabst%
    \setlength\innabst{0.0ex}% Abstand zwischen Text und Zellenrand
    \let\nodeWidth\undefined%
    \newlength\nodeWidth%
    \settowidth{\nodeWidth}{\pgfinterruptpicture\footnotesize#2\endpgfinterruptpicture}%
    \setlength\nodeWidth{\nodeWidth+\innabst+\innabst}%
    \ifnumcomp{\numexpr#1\relax}{=}{0}{}{%
        \setcounter{temp}{\numexpr#1-1\relax}%
        \foreach \i in {0,...,\thetemp}{%\numexpr#1-1\relax}{%
            \tikz[baseline]{%
                % \begin{pgfonlayer}{main}%
                    #8%
                % \end{pgfonlayer}%
                %\draw[brown](current bounding box.south west)rectangle(current bounding box.north east);
            }%
        }%
    }%
    \ifnumcomp{\numexpr#1\relax}{<}{\numexpr#2\relax}{%
        \setcounter{temp}{\numexpr#2-#1-1\relax}%
        \foreach \i in {\thetemp,...,0}{%{\numexpr#2-#1-1\relax,...,0}{%
            \tikz[baseline]{%
                % \begin{pgfonlayer}{main}%
                    #9%
                % \end{pgfonlayer}%
            }%
        }%
    }{}%
    % - - - - - -
    % \ifnumcomp{\numexpr#1\relax}{=}{0}{}{%
    % \foreach \i in {0,...,\numexpr#1-1\relax}%
    % {%
    % \tikz[baseline]%
    % #8%
    % }%
    % }%
    % \ifnumcomp{\numexpr#1\relax}{<}{\numexpr#2\relax}{%
    % \foreach \i in {\numexpr#2-#1-1\relax,...,0}%
    % {%
    % \tikz[baseline]%
    % #9%
    % }%
    % }{}%
    \hspace*{0.75em}%
    \lipsaroundCol{\LARGE\textcolor{ratingColUsed}{\raisebox{\baselineRaiseLARGE}{\textbf{\sfrac{#1}{#2}}}}}{ratingColUsed}{emptyCol!50}
}%
\newcounter{ratingCustomizedResultLevel}%
%
\newcommand\ratingCustomizedNeedNodeSwitchingHub[9]{%
    \ifnumcomp{\value{draftPhase_rating}}{=}{0}{%
        \ratingCustomizedNeedNode{#1}{#2}{#3}{#4}{#5}{#6}{#7}{#8}{#9}%
    }{%
        \ifnumcomp{\value{draftPhase_rating}}{=}{1}{%
            \textbf{\sfrac{#1}{#2}} -- (\textbf{»Draft«})%
        }{%
            »Invalid Value on \enquote{draftPhase} / \enquote{draftPhase\_rating}«%
        }%
    }%
}
%^^^^^^^^^^^^^^^^^^^^^^^^^^^^^^^^^^^^^^^^^^^^^^^^^^^^^^^^^^^^^^^^^^^^
%------------- Customizable Rating - Most versatile End -------------
%====================================================================
%####################################################################
%
%
%
%
%
%####################################################################
%====================================================================
%------------- Customizable Rating -------------
%vvvvvvvvvvvvvvvvvvvvvvvvvvvvvvvvvvvvvvvvvvvvvvvvvvvvvvvvvvvvvvvvvvvv
%____________________________________________________________________
%ARGS: RatingPoints, RatingOutOf, InverseRating, FilledColor, EmptyColor, filledTextCol, emptyTextCol, shape[tikzset/tikzstyle]
%     In case the InverseRating is given with 1 instead of 0 it is considered as "less points is better", which is then reflected in the coloring of the Rating-Ellipse
\newcommand\ratingCustomized[8]{%
    \ratingCustomizedNeedNodeSwitchingHub{#1}{#2}{#3}{#4}{#5}{#6}{#7}{%
        \begin{pgfonlayer}{main}%
            \node[anchor=base,inner sep=\innabst,%
                minimum width=\nodeWidth,%
                draw,fill=filledCol,text=filledTextCol,%
                #8%
            ]{\raisebox{\baselineRaisefootnotesize}{\footnotesize\number\numexpr\i+1\relax}};%
        \end{pgfonlayer}%
    }{%
        \begin{pgfonlayer}{main}%
            \node[anchor=base,inner sep=\innabst,%
                minimum width=\nodeWidth,%
                draw,dotted,fill=emptyCol,text=emptyTextCol,%
                #8%
            ]{\raisebox{\baselineRaisefootnotesize}{\footnotesize\number\numexpr#2-\i\relax}};%
        \end{pgfonlayer}%
    }
}%
%####################################################################
%====================================================================
%ARGS: RatingPoints, RatingOutOf, RatingInverse, FilledColor, EmptyColor
\newcommand\ratingStarCustomized[7]{%
    \tikzstyle{ratingShapeStar}=[shape=star,star points=5,star point ratio=2]%
    \ratingCustomized{#1}{#2}{#3}{#4}{#5}{#6}{#7}{ratingShapeStar}%
}
\newcommand\ratingDiamondCustomized[7]{%
    \tikzstyle{ratingShapeDiamond}=[diamond,aspect=1,text centered,align=center,inner sep=0.25ex]%
    \ratingCustomized{#1}{#2}{#3}{#4}{#5}{#6}{#7}{ratingShapeDiamond}%
}
%^^^^^^^^^^^^^^^^^^^^^^^^^^^^^^^^^^^^^^^^^^^^^^^^^^^^^^^^^^^^^^^^^^^^
%------------- Customizable Rating End -------------
%====================================================================
%####################################################################
%
%
%
%
%
%####################################################################
%====================================================================
%------------- Customizable Rating -------------
%vvvvvvvvvvvvvvvvvvvvvvvvvvvvvvvvvvvvvvvvvvvvvvvvvvvvvvvvvvvvvvvvvvvv
%____________________________________________________________________
\newcommand\rating[2]{%
    \definecolor{filledStarCol}{named}{yellow}%
    \definecolor{emptyStarCol}{RGB}{240,240,220}%
    \ratingStarCustomized{#1}{#2}{0}{filledStarCol}{emptyStarCol}{black}{black}%
}%
%Rating about the Time required to put in
\newcommand\timerating[2]{%
    \definecolor{filledClockCol}{RGB}{120,30,220}%{150,50,250}%
    \definecolor{emptyClockCol}{RGB}{210,190,230}%
    \definecolor{filledClockTextCol}{RGB}{200,100,0}%
    %
    \definecolor{ZeigerColOne}{RGB}{20,20,20}%
    \definecolor{ZeigerColTwo}{RGB}{180,180,180}%
    %ARGS: RatingPoints, RatingOutOf, InverseRating, FilledColor, EmptyColor, filledTextCol, emptyTextCol, NodeToDrawFilled, NodeToDrawEmpty
    \ratingCustomizedNeedNodeSwitchingHub{#1}{#2}{0}{filledClockCol}{emptyClockCol}{filledClockTextCol}{black}{%
        \drawClockTikz{#2}{\raisebox{\baselineRaisefootnotesize}{\footnotesize\number\numexpr\i+1\relax}}{filledCol}{white}{ZeigerColOne}{}%
    }{%
        \drawClockTikz{#2}{\raisebox{\baselineRaisefootnotesize}{\footnotesize\number\numexpr#2-\i\relax}}{emptyClockCol}{black}{ZeigerColTwo}{dotted}%
        % \begin{pgfonlayer}{Layer1}
        % \node[anchor=base,inner sep=\innabst,%
        %     minimum width=\circleWidth,%
        %     ratingShapeStar,
        %     draw,dotted,fill=emptyCol,text=emptyTextCol%
        % ]{\footnotesize\number\numexpr#2-\i\relax};%
        % \end{pgfonlayer}
    }%
}%
%Quantification, Classification
\newcommand\quant[2]{%
    \definecolor{filledDiamondCol}{RGB}{80,80,255}%
    \definecolor{emptyDiamondCol}{RGB}{200,200,255}%
    \ratingDiamondCustomized{#1}{#2}{0}{filledDiamondCol}{emptyDiamondCol}{white}{black}%
}%
%====================================================================
%####################################################################
%Args: MostLongContent, Label, FillColor, LabelColor, ZeigerColor, AdditionalStyleOptions
\newcommand\drawClockTikz[6]{%
    \let\tmpLen\undefined%
    \newlength\tmpLen%
    %
    \let\innAbst\undefined%
    \newlength\innAbst%
    \setlength\innAbst{0.2ex}% Abstand zwischen Text und Zellenrand
    %
    \let\circleWidth\undefined%
    \newlength\circleWidth%
    \setlength\circleWidth{0ex}% Abstand zwischen Text und Zellenrand
    \settowidth\circleWidth{\pgfinterruptpicture\footnotesize#1\endpgfinterruptpicture}%\footnotesize\number\numexpr#2\relax}%
    \setlength\circleWidth{\circleWidth+2\innAbst}%
    %
    \let\circleHeight\undefined%
    \newlength\circleHeight%
    \setlength\circleHeight{0ex}% Abstand zwischen Text und Zellenrand
    \settoheight\circleHeight{\pgfinterruptpicture\footnotesize#1\endpgfinterruptpicture}%
    \setlength\circleHeight{\circleHeight+2\innAbst}%
    %
    \let\circleRadius\undefined%
    \newlength\circleRadius%
    % \ifthenelse{\circleHeight>\circleWidth}{\setlength\circleRadius{\circleHeight}}{\setlength\circleRadius{\circleHWidth}}%
    % \setlength\circleRadius{\circleRadius/2}%
    %
    \setlength\tmpLen{1.5em}%
    \ifthenelse{\tmpLen>\circleHeight}{\setlength\circleHeight{\tmpLen}}{}%
    %
    \tikzstyle{clockZeiger}=[draw=#5,line width=0.3ex]%draw=colorBoxBorder,fill=colorBoxBG
    %
    \begin{pgfonlayer}{Layer5}%Label % serves as first anchor
        \node[anchor=base,inner sep=\innAbst,%
            %draw,fill=filledCol,%text opacity=0,%
            minimum width=\circleWidth,minimum height=\circleHeight,%
            text=#4,%
            circle](label\i)%
            {#2};%\footnotesize\number\numexpr\i+1\relax};%
    \end{pgfonlayer}
    \begin{pgfonlayer}{Layer1}%Uhr Base-Shape & Füllung 1 % serves as 'new' anchor for everything else upcoming
        \node[anchor=center,inner sep=\innAbst,%
            %draw,
            fill=#3,%text opacity=0,%
            minimum width=\circleWidth,minimum height=\circleHeight,%
            circle](\i)at(label\i.center)%
            {};%\footnotesize\number\numexpr\i+1\relax};%
    \end{pgfonlayer}
        \coordinate(A)at(\i.center);
        \coordinate(B)at(\i.north);
        \calcLength(A,B){tmpLen}%
        \circleRadius=\tmpLen pt%
    \begin{pgfonlayer}{Layer4}
        \path[clockZeiger](\i.center)--($(\i.center)+(90:\circleRadius+0.6ex)$);%
        \path[clockZeiger](\i.center)--($(\i.center)+(-20:\circleRadius+0.6ex)$);%
    \end{pgfonlayer}
    \begin{pgfonlayer}{Layer2}%Hellere Uhrenfüllung
        \path[fill=#3!50](\i.center)--($(\i.center)+(-20:\circleRadius)$)arc(-20:-270:\circleRadius)--(\i.center)--cycle;%
    \end{pgfonlayer}
    \begin{pgfonlayer}{Layer3}%UhrenRahmen & Striche
        \let\outerRingAddR\undefined%
        \newlength\outerRingAddR%
        \setlength\outerRingAddR{0.15em}%
        \path[draw,#6]($(\i.north)+(0,\outerRingAddR)$)arc(90:430:\circleRadius+\outerRingAddR)--cycle;%
        \let\strokeWidth\undefined%
        \newlength\strokeWidth%
        \setlength\strokeWidth{0.175ex}%
        \foreach \k in {0,30,...,360}{%
            \path[draw=#5!50,line width=\strokeWidth,#6]($(\i.center)+(\k:0.85\circleRadius)$)--($(\i.center)+(\k:\circleRadius+\outerRingAddR)$);
        }%
    \end{pgfonlayer}
}%
%^^^^^^^^^^^^^^^^^^^^^^^^^^^^^^^^^^^^^^^^^^^^^^^^^^^^^^^^^^^^^^^^^^^^
%------------- Customizable Rating End -------------
%====================================================================
%####################################################################
%
%
%
%
%
%________________________________________________________________________
%------------------------------------------------------------------------
%						For my own Macros (like the Recipe Book Formatting)
%						For setting the Draft-State ("draftPhase") (See also the File settings.tex)
%/\/\/\/\/\/\/\/\/\/\/\/\/\/\/\/\/\/\/\/\/\/\/\/\/\/\/\/\/\/\/\/\/\/\/\/\
% Used by the Macros formatting the Recipe-Book
% - 0: Final Version. Everything formats nice and fine and as appealing as intended
% - 1: Works as a draft state to save compiling time during working on it. Pictures are not included and the Rating-Symbols are not generated
\newcounter{draftPhase}%
% These two are the ones actually used by the Macros. Can be selectively set to new values after this position here to let their behaviour diverge
\newcounter{draftPhase_pic}%
\newcounter{draftPhase_rating}%
\newcounter{draftPhase_ingredients}%
% = = = = = = = = = = = = = = = = = = = = = = = =
\newcommand\draftPhaseSetGlobal[1]{%
    \setcounter{draftPhase}{#1}%
    \setcounter{draftPhase_pic}{\value{draftPhase}}%
    \setcounter{draftPhase_rating}{\value{draftPhase}}%
    \setcounter{draftPhase_ingredients}{\value{draftPhase}}%
}%Sets the values of all
% These two are the ones actually used by the Macros. Can be selectively set to new values after this position here to let their behaviour diverge
\newcommand\draftPhaseSetPicture[1]{%
    \setcounter{draftPhase_pic}{#1}%
}% Sets only and selectively the value used for Picture included during the \dishPicMat-Macro
\newcommand\draftPhaseSetRating[1]{%
    \setcounter{draftPhase_rating}{#1}%
}% Sets only and selectively the value used for generating the rating-symbols
\newcommand\draftPhaseSetIngredients[1]{%
    \setcounter{draftPhase_ingredients}{#1}%
}% Sets only and selectively the value used for generating the rating-symbols
%/\/\/\/\/\/\/\/\/\/\/\/\/\/\/\/\/\/\/\/\/\/\/\/\/\/\/\/\/\/\/\/\/\/\/\/\
%							Settings for my own Macros end
%------------------------------------------------------------------------
%________________________________________________________________________
% 
% 
% 
% 
% 
%
%####################################################################
%====================================================================
%------------- A Command / Environment for Displaying a whole Rating-Section -------------
%vvvvvvvvvvvvvvvvvvvvvvvvvvvvvvvvvvvvvvvvvvvvvvvvvvvvvvvvvvvvvvvvvvvv
%____________________________________________________________________
%ARGS: The ratings: Köstlichkeit, Aufwand, Gesund
%    Optional: Als erstes Argument, wie viele Argumente übergeben werden. Default 2. Bei 3, wird also der Wert für "Gesund" auch noch verarbeitet
\DeclareDocumentCommand{\ratingSection}{%
O{2} m m m%
}{%
    \begin{longtable}[l]{|l l l |}%@{}l
    \hline
    Köstlichkeit:&%
        #2&%
        \ratingSectionDescriptionFormat{Mehr Punkte}{Besser}%
        \TstrutRatingSection\BstrutRatingSection\\
    \hline
    Aufwand:&%
        #3&%
        \ratingSectionDescriptionFormat{Mehr Punkte}{Weniger Aufwand}%
        \TstrutRatingSection\BstrutRatingSection\\
    \hline
    \ifnumcomp{\numexpr#1\relax}{>}{2}{%
    Gesundheit:&%
        #4&%
        \ratingSectionDescriptionFormat{Mehr Punkte}{Gesünder}%
        \TstrutRatingSection\BstrutRatingSection\\
    \hline
    }{}%
    \end{longtable}%
}%
\newcommand{\ratingSectionDescriptionFormat}[2]{%ARGS: #1: Something like "Mehr Punkte". #2: What the first Expressions means as a result, like "Weniger Aufwand"
    \makecell[l]{\color{black!50}\footnotesize (#1$\rightarrow$\\\color{black!50}\footnotesize\ \ #2)}%
}%
%====================================================================
% define "struts", as suggested by Claudio Beccari in a piece in TeX and TUG News, Vol. 2, 1993.
\newcommand\TstrutRatingSection{\rule{0pt}{4ex}}         % = `top' strut
\newcommand\BstrutRatingSection{\rule[-2.6ex]{0pt}{0pt}}   % = `bottom' strut
%^^^^^^^^^^^^^^^^^^^^^^^^^^^^^^^^^^^^^^^^^^^^^^^^^^^^^^^^^^^^^^^^^^^^
%------------- Rating Section End -------------
%====================================================================
%####################################################################
% 
% 
% 
% 
% 
%
%####################################################################
%====================================================================
%------------- Customized Inclusions -------------
%vvvvvvvvvvvvvvvvvvvvvvvvvvvvvvvvvvvvvvvvvvvvvvvvvvvvvvvvvvvvvvvvvvvv
%____________________________________________________________________
%If this is set to '0', Recipes using the Inclusion Macro are included like normal. Set to '1', they are not included
\newcounter{draftPhase_SurpressRecipes}%
\newcommand{\draftPhaseSetSurpressRecipes}[1]{%
    \setcounter{draftPhase_SurpressRecipes}{#1}%
}%
% - - - - - - - - - - - - -
% For including Recipes & be able to disable the Inclusion, while overriding the exclusion on individual occasions
%ARGS:
% - [Optional](#1) - If the exclusion shall be 'bypassed'. If this is set 'notAtAll(not present optional argument)' or to zero (or anything other than '1'), the Inclusion of the Recipe depends on the global Setting of "\draftPhaseSurpressRecipes". If a '1' is passed, this Recipe is included, compiled and displayed in any way, disregarding the global setting.
% - Mandatory(#2) - The path of the Recipe to include
\DeclareDocumentCommand{\inputRecipe}{%
O{0} m%
}{%
    \ifnumcomp{#1}{=}{1}{%
        \input{#2}%
    }{%
        \ifnumcomp{\value{draftPhase_SurpressRecipes}}{=}{0}{%
            \input{#2}%
        }{%
            \ifnumcomp{\value{draftPhase_SurpressRecipes}}{=}{1}{%
            }{%
                »Invalid Value on \enquote{draftPhase\_SurpressRecipes}«%
            }%
        }%
    }%
}%
%^^^^^^^^^^^^^^^^^^^^^^^^^^^^^^^^^^^^^^^^^^^^^^^^^^^^^^^^^^^^^^^^^^^^
%------------- Inclusions End -------------
%====================================================================
%####################################################################
%