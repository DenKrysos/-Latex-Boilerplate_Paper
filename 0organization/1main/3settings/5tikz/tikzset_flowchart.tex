%
\tikzset{%
    o/.style={%
        shorten >=0,%
        decoration={%
            markings,%
            mark={%
                at position 1%
                with {%
                    \fill circle [radius=#1];%
                }%
            }%
        },%
        postaction=decorate%
    },%
    o/.default=0.5*\pgflinewidth%
}%
%
\definecolor{muxpurple}{RGB}{210,140,210}%
\definecolor{entryred}{RGB}{255,50,50}%
\definecolor{exitgreen}{RGB}{70,200,80}%
\definecolor{slightblue}{RGB}{230,230,255}%
%
\tikzstyle{startstop}=[%
	inner sep=\innabst,%
	rectangle,%
	rounded corners,%
	minimum width=3cm,%
	minimum height=1cm,%
	text centered,%
	align=center,%
	draw=black,%
	fill=red!30%
]%
\tikzstyle{io}=[%
	trapezium,%
	trapezium left angle=70,%
	trapezium right angle=110,%
	minimum width=2cm,%
	minimum height=1cm,%
	inner ysep=\innabst,%
	inner xsep=0.15*\innabst,%
	text centered,%
	align=center,%
	draw=black,%
	fill=blue!30%
]%
\tikzstyle{interthread}=[%
	trapezium,%
	trapezium left angle=-70,%
	trapezium right angle=-110,%
	minimum width=2cm,%
	minimum height=1cm,%
	inner ysep=\innabst,%
	inner xsep=0.15*\innabst,%
	text centered,%
	align=center,%
	draw=black,%
	fill=cyan!30!white!90!black%
]%
\tikzstyle{processsplitl}=[%
	minimum width=3cm,%
	%minimum height=1cm,%
	align=left,%
	inner sep=0.5*\innabst,%
	draw=black,%
	rectangle split,%
	rectangle split parts=#1,%
	rectangle split part fill={orange!30,orange!30},%
    rectangle split every empty part={},% delete existing height, depth and width
	rectangle split empty part height=0.1ex,%
	rectangle split empty part depth=0ex,%
	rectangle split part align=left,%either: top, center, base, bottom, left, right
%			You could do something like this, to do differing aligns for multiple parts:
%				rectangle split part align={center, left, left, left},%
% 	shade,%
% 	shading=axis,%
% 	top color=white,%
% 	bottom color=processblue,%
% 	shading angle=45,%
]%
\tikzstyle{processsplit}=[%
	processsplitl=#1,%
	rectangle split part align=center,%
	text centered,%
	align=center,%
]%
\tikzstyle{process}=[%
	rectangle,%
	minimum width=2cm,%
	minimum height=1cm,%
	text centered,%
	align=center,%
	draw=black,%
	fill=orange!30%
]%
\tikzstyle{decision}=[%
	diamond,%
	aspect=3,%
	minimum width=3cm,%
	minimum height=1cm,%
	text centered,%
	align=center,%
	draw=black,%
	fill=green!30%
]%
\tikzstyle{decision2}=[%
	decision,%
	diamondcut,%
	aspect=1,%
	minimum width=0cm,%
	minimum height=0cm,%
	minimum size=0cm,%
	inner sep=0.25ex,%
]%
\tikzstyle{mux}=[%
	trapezium,%
	trapezium left angle=110,%
	trapezium right angle=110,%
	minimum width=3cm,%
	minimum height=1cm,%
	inner sep=\innabst,%
% 	inner sep=0pt,%
	text centered,%
	align=center,%
	draw=black,%
	fill=muxpurple%
]%
\newcommand{\muxwidth}[1]{#1*3cm+2*0.36cm}%
\tikzstyle{mux2}=[%
	trapezium,%
	trapezium left angle=-70,%
	trapezium right angle=110,%
	trapezium stretches=true,%
	trapezium stretches body=true,%
	minimum size=0.0cm,%
	minimum width=#1*3cm+#1*\pgflinewidth+0.6cm,%
	minimum height=1.0cm,%
	inner sep=\innabst,%
% 	inner xsep=0.0cm,%
% 	outer sep=0.0cm,%
% 	text height=1.0cm,%
% 	text width=6.0cm,%
	text centered,%
	align=center,%
	draw=black,%
	fill=muxpurple%
]%
\tikzstyle{muxwahl}=[%
	regular polygon,%
	regular polygon sides=3,%
	shape border rotate=180,%
	yscale=0.35,%
	minimum width=3.5cm,%
	minimum height=0.1cm,%
	inner sep=0.0cm,%
	text centered,%
	align=center,%
	draw=black,%
	fill=muxpurple%
]%
\tikzstyle{muxwahl2}=[%
	trapezium,%
	trapezium left angle=-45,%
	trapezium right angle=-45,%
	minimum width=3cm,%
% 	text width=3cm,%
	minimum height=1cm,%
	inner sep=\innabst,%
% 	inner sep=0pt,%
	text centered,%
	align=center,%
	draw=black,%
	fill=muxpurple%
]%
\tikzstyle{note}=[%
	rectangle,%
	minimum width=2cm,%
	minimum height=1cm,%
	text centered,%
	align=center,%
	draw=black,%
	dashed,%
	rounded corners,%
]%
\tikzstyle{cleartext}=[%
	solid,%
	align=center,%
	inner sep=0pt,%
	minimum width=0pt,%
	minimum height=0pt,%
]%
\tikzstyle{arrow}=[ultra thick,->,>=stealth]%
\tikzstyle{arrowdouble}=[ultra thick,<->,>=latex]%
\tikzstyle{arrowwait}=[ultra thick,->,>=stealth,
	dash pattern=on \pgflinewidth off 6pt]%
\tikzstyle{arrowkomm}=[ultra thick,<->,>=latex,
	dash pattern=on 8pt off 6pt on 2pt off 6pt]%
\tikzstyle{connect}=[ultra thick,-,o]%
\tikzstyle{vecarrow}=[thick, decoration={markings,mark=at position%
   1 with {\arrow[semithick]{open triangle 60}}},%
   double distance=1.4pt, shorten >= 5.5pt,%
   preaction = {decorate},%
   postaction = {draw,line width=1.4pt, white,shorten >= 4.5pt}]%
\definecolor{Datastream_Color}{RGB}{100,0,170}%
\definecolor{MPTCP_Path1_Color}{RGB}{255,60,0}%
\definecolor{MPTCP_Path2_Color}{RGB}{50,0,255}%
\tikzstyle{cable}=[ultra thick,double,double distance=4pt,%
% 	I,%
	shorten >= -0.1pt,%
	postaction = {draw,line width=4pt, white, shorten >= -0.2pt},%
	postaction = {draw,line width=1pt, Datastream_Color, shorten >= -0.2pt}]%
\tikzstyle{cable2}=[cable,%
	postaction = {draw=MPTCP_Path2_Color,line width=1pt, shorten >= -0.2pt}]%
\tikzstyle{cable_optional}=[cable,%
	postaction = {draw=MPTCP_Path1_Color,line width=1pt, shorten >= -0.2pt}]%
\tikzstyle{innerWhite}=[semithick, white,line width=1.4pt, shorten >= 4.5pt]%
%
%
%
%-----------------------------------------------------------
% For Finite State Machines (FSM)
%-----------------------------------------------------------
%========-------------------------------------------========
%========		First some specials					========
%========-------------------------------------------========
% They are for "`nice"' double border nodes
\tikzset{doubleA/.style = {matrix of nodes,%
    draw, inner sep=1mm,%
    nodes = {rectangle, draw, inner sep=.3333em}}}%
%
\tikzset{doubleB/.style = {circle, draw, %
    append after command={%
        \pgfextra{\node[fit=(\tikzlastnode),%
        	circle,%
        	draw,%
        	minimum size=0pt,%
        	minimum height=0pt,%
        	minimum width=0pt,%
        	inner sep=0pt]%
        	(\tikzlastnode-out){};%
        }%
    }%
    }}%
% Example for use of doubleA and doubleB:
%	Google Drive\Bibliothek\Latex\tikz\Beispiele,Templates,usw\doubleAdoubleB-Bsp.tex
% ========-------------------------------------------========
\tikzstyle{fsmstart2}=[%
	circle,%
	draw=black,%
	line width=1.5pt,%
	double=white,%
	double distance=0.5ex,%
	inner sep=1ex,%
	text centered,%
	align=center,%
]%
% Alternative. Gives both the Anchors on the outer and the inner border
\tikzstyle{fsmstart}=[%
	doubleB,%
	inner sep=1.0ex,%
]%
\tikzstyle{fsmfinish}=[%
	circle,%
	draw=black,%
	inner sep=1.5ex,%
]%
\tikzstyle{fsmstate}=[%
	draw,%
	shape=ellipse,%
	anchor=base,%
	inner sep=1.0ex,%
	text centered,%
	align=center,%
]%
%
%
% use the "`bend"' arrows with (Bsp.)
%	\draw[arrowoben](sourcenode)to(destinationnode)
%		(instead of) (look between the node coordinates)
%	\draw[arrowoben](sourcenode)--(destinationnode)
\tikzstyle{arrowoben}=[%
	arrow,%
	shorten >=1ex,%
	shorten <=1ex,%
	bend angle=30,%
	bend left,%
]%
\tikzstyle{arrowunten}=[%
	arrowoben,%
	bend right,%
]%
%
%